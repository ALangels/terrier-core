{[}\href{website_search.html}{Previous: Website Search Application}{]}
{[}\href{index.html}{Contents}{]}
{[}\href{hadoop_configuration.html}{Next: Terrier/Hadoop
Configuration}{]}\\

\section{Examples of using Terrier to index TREC collections: WT2G \&
Blogs06}\label{examples-of-using-terrier-to-index-trec-collections-wt2g-blogs06}

Terrier can index all known TREC test collections. We refer readers to
the \href{http://ir.dcs.gla.ac.uk/wiki/Terrier}{Terrier wiki} for latest
configuration for indexing various collections:

\begin{itemize}
\tightlist
\item
  \href{http://ir.dcs.gla.ac.uk/wiki/Terrier/Disks1\&2}{Disks1\&2}
\item
  \href{http://ir.dcs.gla.ac.uk/wiki/Terrier/Disks4\&5}{Disks4\&5}
\item
  \href{http://ir.dcs.gla.ac.uk/wiki/Terrier/WT2G}{WT2G}
\item
  \href{http://ir.dcs.gla.ac.uk/wiki/Terrier/WT10G}{WT10G}
\item
  \href{http://ir.dcs.gla.ac.uk/wiki/Terrier/DOTGOV}{DOTGOV} -- see also
  the \href{learning.html}{learning-to-rank} documentation page.
\item
  \href{http://ir.dcs.gla.ac.uk/wiki/Terrier/DOTGOV2}{DOTGOV2}
\item
  \href{http://ir.dcs.gla.ac.uk/wiki/Terrier/Blogs06}{Blogs06}
\item
  \href{http://ir.dcs.gla.ac.uk/wiki/Terrier/Blogs08}{Blogs08}
\item
  \href{http://ir.dcs.gla.ac.uk/wiki/Terrier/ClueWeb09-B}{ClueWeb09-B}
\item
  \href{http://ir.dcs.gla.ac.uk/wiki/Terrier/ClueWeb12}{ClueWeb12}
\end{itemize}

\subsection{TREC WT2G Collection}\label{trec-wt2g-collection}

Here we give an example of using Terrier to index WT2G - a standard
\href{http://trec.nist.gov}{TREC} test collection. We assume that the
operating system is Linux, and that the collection, along with the
topics and the relevance assessments, is stored in the directory
\texttt{/local/collections/WT2G}. The following configurations are
sufficient for batch retrieval, however if you want to build a web-based
search interface for searching WT2G, see
\href{terrier_http.html}{Web-based Terrier}.

\begin{verbatim}
#goto the terrier folder
cd terrier

#get terrier setup for using a trec collection
bin/trec_setup.sh /local/collections/WT2G/

#rebuild the collection.spec file correctly
find /local/collections/WT2G/ -type f | sort |grep -v info > etc/collection.spec

#use In_expB2 DFR model for querying
echo trec.model=org.terrier.matching.models.In_expB2 >> etc/terrier.properties

#use this file for the topics
echo trec.topics=/local/collections2/WT2G/info/topics.401-450.gz >> etc/terrier.properties

#use this file for query relevance assessments
echo trec.qrels=/local/collections2/WT2G/info/qrels.trec8.small_web.gz >> etc/terrier.properties

#index the collection
bin/trec_terrier.sh -i

#run the topics, with suggested c value 10.99 
bin/trec_terrier.sh -r -c 10.99
#run topics again with query expansion enabled
bin/trec_terrier.sh -r -q -c 10.99

#evaluate the results in var/results/
bin/trec_terrier.sh -e

#display the Mean Average Precision
tail -1 var/results/*.eval
#MAP should be 
#In_expB2 Average Precision: 0.3160 
\end{verbatim}

\subsection{TREC Blogs06 Collection}\label{trec-blogs06-collection}

This guide will provide a step-by-step example on how to use Terrier for
indexing, retrieval and evaluation. We use TREC Blogs06 test collection,
along with the corresponding topics and the qrels from TREC 2006 Blog
track. We assume that these are stored in the directory
\texttt{/local/collections/Blogs06/}

\subsubsection{Indexing}\label{indexing}

In the Terrier folder, use trec\_setup.sh to generate a collection.spec
for indexing the collection:

\begin{verbatim}
[user@machine terrier]$ ./bin/trec_setup.sh /local/collections/Blogs06/
[user@machine terrier]$ find /local/collections/Blogs06/ -type f  
    | grep 'permalinks-' |sort > etc/collection.spec
\end{verbatim}

This will result in the creation of a \texttt{collection.spec} file, in
the \texttt{etc} directory, containing a list of the files in the
\texttt{/local/collections/Blog06/} directory. At this stage, you should
check the \texttt{etc/collection.spec}, to ensure that it only contains
files that should be indexed, and that they are sorted (ie
\texttt{20051206/permalinks-000.gz} is the first file).

The TREC Blogs06 collection differs from other TREC collections in that
not all tags should be indexed. For this reason, you should configure
the parse in TRECCollection not to process these tags. Set the following
properties in your \texttt{etc/terrier.properties} file:

\begin{verbatim}
TrecDocTags.doctag=DOC
TrecDocTags.idtag=DOCNO
TrecDocTags.skip=DOCHDR,DATE_XML,FEEDNO,BLOGHPNO,BLOGHPURL,PERMALINK
\end{verbatim}

Finally, the length of the DOCNOs in the TREC Blogs06 collection are 31
characters, longer than the default 20 characters in Terrier. To deal
with this, update properties relating to the MetaIndex in
terrier.properties:

\begin{verbatim}
indexer.meta.forward.keys=docno
indexer.meta.forward.keylens=31
indexer.meta.reverse.keys=docno
\end{verbatim}

Now you are ready to start indexing the collection.

\begin{verbatim}
[user@machine terrier]$ ./bin/trec_terrier.sh -i
Setting TERRIER_HOME to /local/terrier
INFO - TRECCollection read collection specification
INFO - Processing /local/collections/Blogs06/20051206/permalinks-000.gz
INFO - creating the data structures data_1
INFO - Processing /local/collections/Blogs06/20051206/permalinks-001.gz
INFO - Processing /local/collections/Blogs06/20051206/permalinks-002.gz
<snip>
\end{verbatim}

If we did not plan to use Query Expansion initially, then the faster
single-pass indexing could be enabled, using the -j option of
TrecTerrier. If we decide to use query expansion later, we can use the
\href{javadoc/org/terrier/structures/indexing/singlepass/Inverted2DirectIndexBuilder.html}{Inverted2DirectIndexBuilder}
to create the direct index
(\href{javadoc/org/terrier/structures/indexing/singlepass/BlockInverted2DirectIndexBuilder.html}{BlockInverted2DirectIndexBuilder}
for blocks).

\begin{verbatim}
[user@machine terrier]$ ./bin/trec_terrier.sh -i -j
Setting TERRIER_HOME to /local/terrier
INFO - TRECCollection read collection specification
INFO - Processing /local/collections/Blogs06/20051206/permalinks-000.gz
Starting building the inverted file...
INFO - creating the data structures data_1
INFO - Creating IF (no direct file)..
INFO - Processing /local/collections/Blogs06/20051206/permalinks-001.gz
INFO - Processing /local/collections/Blogs06/20051206/permalinks-002.gz
<snip>
[user@machine terrier]$ ./bin/anyclass.sh org.terrier.structures.indexing.singlepass.Inverted2DirectIndexBuilder
INFO - Generating a direct index from an inverted index
INFO - Iteration - 1 of 20 iterations
INFO - Generating postings for documents with ids 0 to 120435
INFO - Writing the postings to disk
<snip>
INFO - Finishing up: rewriting document index
INFO - Finished generating a direct index from an inverted index
\end{verbatim}

Indexing will take a reasonable amount of time on a modern machine.
Additionally, expect to double indexing time if block indexing is
enabled. Using single-pass indexing is significantly faster, even if the
direct file has to be built later.

\subsubsection{Retrieval}\label{retrieval}

Once the index is built, we can do retrieval using the index, following
the steps described below.

First, tell Terrier the location of the topics and relevance assessments
(qrels).

\begin{verbatim}
[user@machine terrier]$ echo trec.topics=/local/collections/Blog06/06.topics.851-900 >> etc/terrier.properties
[user@machine terrier]$ echo trec.qrels=/local/collections/Blog06/qrels.blog06 >> etc/terrier.properties
\end{verbatim}

Next, we should specify the retrieval weighting model that we want to
use. In this case we will use the DFR model called PL2 for ranking
documents (blog posts).

\begin{verbatim}
echo trec.model=org.terrier.matching.models.PL2 >> etc/terrier.properties
\end{verbatim}

Now we are ready to start retrieval. We use the \texttt{-c} to set the
parameter of the weighting model to the value 1. Terrier will do
retrieval by taking each query (called a topic) from the specified
topics file, query the index using it, and save the results to a file in
the \texttt{var/results} folder, named similar to
\texttt{PL2c1.0\_0.res}. The file \texttt{PL2c1.0\_0.res.settings}
contains a dump of the properties and other settings used to generated
the run.

\begin{verbatim}
[user@machine terrier]$ ./bin/trec_terrier.sh -r -c 1
Setting TERRIER_HOME to /local/terrier
INFO - 900 : mcdonalds
INFO - Processing query: 900
<snip>
INFO - Finished topics, executed 50 queries in 27 seconds, results written to 
    terrier/var/results/PL2c1.0_0.res
Time elapsed: 40.57 seconds.
\end{verbatim}

\subsubsection{Evaluation}\label{evaluation}

We can now evaluate the retrieval performance of the generated run using
the qrels specified earlier:

\begin{verbatim}
[user@machine terrier]$ ./bin/trec_terrier.sh -e
Setting TERRIER_HOME to /local/terrier
INFO - Evaluating result file: /local/terrier/var/results/PL2c1.0_0.res
Average Precision: 0.2703
Time elapsed: 3.177 seconds.
\end{verbatim}

Note that more evaluation measures are stored in the file
\texttt{var/results/PL2c1.0\_0.eval}.

\href{}{}

\subsection{Common TREC Settings}\label{common-trec-settings}

This page provides examples of settings for indexing and retrieval on
TREC collections. For example, to index the disk1\&2 collection, the
\texttt{etc/terrier.properties} should look like as follows:

\begin{verbatim}

#default controls for query expansion
querying.postprocesses.order=QueryExpansion
querying.postprocesses.controls=qe:QueryExpansion

#default and allowed controls
querying.default.controls=c:1.0,start:0,end:999
querying.allowed.controls=c,scope,qe,start,end

matching.retrieved_set_size=1000

#document tags specification
#for processing the contents of
#the documents, ignoring DOCHDR
TrecDocTags.doctag=DOC
TrecDocTags.idtag=DOCNO
TrecDocTags.skip=DOCHDR
#the tags to be indexed
TrecDocTags.process=TEXT,TITLE,HEAD,HL
#do not store position information in the index. Set it to true otherwise.
block.indexing=false

#query tags specification
TrecQueryTags.doctag=TOP
TrecQueryTags.idtag=NUM
TrecQueryTags.process=TOP,NUM,TITLE
TrecQueryTags.skip=DOM,HEAD,SMRY,CON,FAC,DEF,DESC,NARR

#stop-words file. default folder is ./share
stopwords.filename=stopword-list.txt

#the processing stages a term goes through
#the following setting applies standard stopword removal and Porter's stemming algorithm.
termpipelines=Stopwords,PorterStemmer
\end{verbatim}

The following table lists the indexed tags (corresponding to the
property \texttt{TrecDocTags.process}) and the running time for a
singlepass inverted index creation on 6 TREC collections. No indexed
tags are specified for the WT2G, WT10G, DOTGOV and DOTGOV2 collections,
which means the system indexes everything in these collections. The
indexing was done on a CentOS 5 Linux machine with Intel Core2 2.4GHz
CPU and 2GB RAM (a maximum of 1GB RAM is allocated to the Java virtual
machine).

\begin{longtable}[]{@{}lll@{}}
\toprule
\begin{minipage}[t]{0.30\columnwidth}\raggedright\strut
Collection
\strut\end{minipage} &
\begin{minipage}[t]{0.30\columnwidth}\raggedright\strut
Indexed tags (\texttt{TrecDocTags.process})
\strut\end{minipage} &
\begin{minipage}[t]{0.30\columnwidth}\raggedright\strut
Indexing time (seconds)
\strut\end{minipage}\tabularnewline
\begin{minipage}[t]{0.30\columnwidth}\raggedright\strut
disk1\&2
\strut\end{minipage} &
\begin{minipage}[t]{0.30\columnwidth}\raggedright\strut
TEXT,TITLE,HEAD,HL
\strut\end{minipage} &
\begin{minipage}[t]{0.30\columnwidth}\raggedright\strut
766.85
\strut\end{minipage}\tabularnewline
\begin{minipage}[t]{0.30\columnwidth}\raggedright\strut
disk4\&5
\strut\end{minipage} &
\begin{minipage}[t]{0.30\columnwidth}\raggedright\strut
TEXT,H3,DOCTITLE,HEADLINE,TTL
\strut\end{minipage} &
\begin{minipage}[t]{0.30\columnwidth}\raggedright\strut
692.115
\strut\end{minipage}\tabularnewline
\begin{minipage}[t]{0.30\columnwidth}\raggedright\strut
WT2G
\strut\end{minipage} &
\begin{minipage}[t]{0.30\columnwidth}\raggedright\strut
~
\strut\end{minipage} &
\begin{minipage}[t]{0.30\columnwidth}\raggedright\strut
709.906
\strut\end{minipage}\tabularnewline
\begin{minipage}[t]{0.30\columnwidth}\raggedright\strut
WT10G
\strut\end{minipage} &
\begin{minipage}[t]{0.30\columnwidth}\raggedright\strut
~
\strut\end{minipage} &
\begin{minipage}[t]{0.30\columnwidth}\raggedright\strut
3,556.09
\strut\end{minipage}\tabularnewline
\begin{minipage}[t]{0.30\columnwidth}\raggedright\strut
DOTGOV
\strut\end{minipage} &
\begin{minipage}[t]{0.30\columnwidth}\raggedright\strut
~
\strut\end{minipage} &
\begin{minipage}[t]{0.30\columnwidth}\raggedright\strut
4,435.12
\strut\end{minipage}\tabularnewline
\begin{minipage}[t]{0.30\columnwidth}\raggedright\strut
DOTGOV2
\strut\end{minipage} &
\begin{minipage}[t]{0.30\columnwidth}\raggedright\strut
~
\strut\end{minipage} &
\begin{minipage}[t]{0.30\columnwidth}\raggedright\strut
96,340.00
\strut\end{minipage}\tabularnewline
\bottomrule
\end{longtable}

The following table compares the indexing time using the classical
two-phase indexing and single-pass indexing with and without storing the
terms positions (blocks). The table shows that the single-pass indexing
is markedly faster than the two-phase indexing, particular when block
indexing is enabled.

\begin{longtable}[]{@{}lllll@{}}
\toprule
\begin{minipage}[t]{0.17\columnwidth}\raggedright\strut
Collection
\strut\end{minipage} &
\begin{minipage}[t]{0.17\columnwidth}\raggedright\strut
Two-phase
\strut\end{minipage} &
\begin{minipage}[t]{0.17\columnwidth}\raggedright\strut
Single-pass
\strut\end{minipage} &
\begin{minipage}[t]{0.17\columnwidth}\raggedright\strut
Two-phase + blocks
\strut\end{minipage} &
\begin{minipage}[t]{0.17\columnwidth}\raggedright\strut
Single-pass + blocks
\strut\end{minipage}\tabularnewline
\begin{minipage}[t]{0.17\columnwidth}\raggedright\strut
disk1\&2
\strut\end{minipage} &
\begin{minipage}[t]{0.17\columnwidth}\raggedright\strut
13.5 min
\strut\end{minipage} &
\begin{minipage}[t]{0.17\columnwidth}\raggedright\strut
8.65 min
\strut\end{minipage} &
\begin{minipage}[t]{0.17\columnwidth}\raggedright\strut
32.6 min
\strut\end{minipage} &
\begin{minipage}[t]{0.17\columnwidth}\raggedright\strut
12.1 min
\strut\end{minipage}\tabularnewline
\begin{minipage}[t]{0.17\columnwidth}\raggedright\strut
disk4\&5
\strut\end{minipage} &
\begin{minipage}[t]{0.17\columnwidth}\raggedright\strut
11.7 min
\strut\end{minipage} &
\begin{minipage}[t]{0.17\columnwidth}\raggedright\strut
7.63 min
\strut\end{minipage} &
\begin{minipage}[t]{0.17\columnwidth}\raggedright\strut
25.0 min
\strut\end{minipage} &
\begin{minipage}[t]{0.17\columnwidth}\raggedright\strut
10.2 min
\strut\end{minipage}\tabularnewline
\begin{minipage}[t]{0.17\columnwidth}\raggedright\strut
WT2G
\strut\end{minipage} &
\begin{minipage}[t]{0.17\columnwidth}\raggedright\strut
9.95 min
\strut\end{minipage} &
\begin{minipage}[t]{0.17\columnwidth}\raggedright\strut
7.52 min
\strut\end{minipage} &
\begin{minipage}[t]{0.17\columnwidth}\raggedright\strut
23.6 min
\strut\end{minipage} &
\begin{minipage}[t]{0.17\columnwidth}\raggedright\strut
10.8 min
\strut\end{minipage}\tabularnewline
\begin{minipage}[t]{0.17\columnwidth}\raggedright\strut
WT10G
\strut\end{minipage} &
\begin{minipage}[t]{0.17\columnwidth}\raggedright\strut
62.5 min
\strut\end{minipage} &
\begin{minipage}[t]{0.17\columnwidth}\raggedright\strut
34.7 min
\strut\end{minipage} &
\begin{minipage}[t]{0.17\columnwidth}\raggedright\strut
2hour 18min
\strut\end{minipage} &
\begin{minipage}[t]{0.17\columnwidth}\raggedright\strut
53.1 min
\strut\end{minipage}\tabularnewline
\begin{minipage}[t]{0.17\columnwidth}\raggedright\strut
DOTGOV
\strut\end{minipage} &
\begin{minipage}[t]{0.17\columnwidth}\raggedright\strut
71.0min
\strut\end{minipage} &
\begin{minipage}[t]{0.17\columnwidth}\raggedright\strut
47.1min
\strut\end{minipage} &
\begin{minipage}[t]{0.17\columnwidth}\raggedright\strut
2hour 45min
\strut\end{minipage} &
\begin{minipage}[t]{0.17\columnwidth}\raggedright\strut
1hour 11min
\strut\end{minipage}\tabularnewline
\bottomrule
\end{longtable}

\href{}{}

The following table lists the retrieval performance achieved using three
weighting models, namely the Okapi
\href{javadoc/org/terrier/matching/models/BM25.html}{BM25}, DFR
\href{javadoc/org/terrier/matching/models/PL2.html}{PL2} and the new
parameter-free
\href{javadoc/org/terrier/matching/models/DFRee.html}{DFRee} model on a
variety of standard TREC test collections. We provide the best values
for the b and c parameters of BM25 and PL2 respectively, by optimising
MAP using a simulated annealing process. In contrast, DFRee performs
robustly across all collections while it does not require any parameter
tuning or training.

\href{javadoc/org/terrier/matching/models/BM25.html}{BM25}

\href{javadoc/org/terrier/matching/models/PL2.html}{PL2}

\href{javadoc/org/terrier/matching/models/DFRee.html}{DFRee}

Collection and tasks

b value

MAP

c value

MAP

MAP

disk1\&2, TREC1-3 adhoc tasks

0.3277

0.2324

4.607

0.2260

0.2175

disk4\&5, TREC 2004 Robust Track

0.3444

0.2502

9.150

0.2531

0.2485

WT2G, TREC8 small-web task

0.2381

0.3186

26.04

0.3252

0.2829

WT10G, TREC9-10 Web Tracks

0.2505

0.2104

12.33

0.2103

0.2030

DOTGOV, TREC11 Topic-distillation task

0.7228

0.1910

1.280

0.2030

0.1945

DOTGOV2, TREC2004-2006 Terabyte Track adhoc tasks

0.39

0.3046

6.48

0.3097

0.2935

Many of the above TREC collections can be obtained directly from either
\href{http://trec.nist.gov}{TREC (NIST)}, or from the
\href{http://ir.dcs.gla.ac.uk/test_collections/}{University of Glasgow}

{[}\href{website_search.html}{Previous: Website Search Application}{]}
{[}\href{index.html}{Contents}{]}
{[}\href{hadoop_configuration.html}{Next: Terrier/Hadoop
Configuration}{]}

\begin{center}\rule{0.5\linewidth}{\linethickness}\end{center}

Webpage: \url{http://terrier.org}\\
Contact:
\href{mailto:terrier@dcs.gla.ac.uk}{\nolinkurl{terrier@dcs.gla.ac.uk}}\\
\href{http://www.dcs.gla.ac.uk/}{School of Computing Science}\\
Copyright (C) 2004-2015 \href{http://www.gla.ac.uk/}{University of
Glasgow}. All Rights Reserved.
