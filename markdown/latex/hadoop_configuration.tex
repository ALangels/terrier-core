{[}\href{trec_examples.html}{Previous: TREC Experiment Examples}{]}
{[}\href{index.html}{Contents}{]} {[}\href{hadoop_indexing.html}{Next:
Hadoop MapReduce Indexing with Terrier}{]}\\

\section{Configuring Terrier for
Hadoop}\label{configuring-terrier-for-hadoop}

\subsection{Overview}\label{overview}

From version 2.2 onwards, Terrier has supported the Hadoop MapReduce
framework. Currently, Terrier provides
\href{hadoop_indexing.html}{single-pass distributed indexing under
MapReduce}, however, Terrier has been designed to be compatible with
other Hadoop driven functionality. In this document, we describe how to
integrate your Hadoop and Terrier setups. Hadoop is useful because it
allows extremely large-scale operations, using MapReduce technology,
built on a distributed file system. More information can be found about
deploying Hadoop using a cluster of nodes in the
\href{http://hadoop.apache.org/core/docs/current/}{Hadoop Core
documentation}.

\subsection{Pre-requisites}\label{pre-requisites}

Terrier requires a working Hadoop setup, built using a cluster of one or
more machines, currently of Hadoop version 0.20.x. The core \emph{may
not} currently support newer versions of Hadoop due to minor changes in
the Hadoop API. However, should you wish to use a newer version to take
advantage of the numerous bug fixes and efficiency improvements
introduced, we anticipate that the core can be updated to achieve this
(see the
\href{http://ir.dcs.gla.ac.uk/wiki/Terrier/Hadoop}{Terrier/Hadoop wiki
page} for more information on upgrading Hadoop in Terrier). In the
Hadoop Core documentation, we recommend
\href{http://hadoop.apache.org/docs/r0.19.0/quickstart.html}{quickstart}
and
\href{http://hadoop.apache.org/docs/r0.19.0/cluster_setup.html}{cluster
setup} documents. If you do not have a dedicated cluster of machines
with Hadoop running and do not wish to create one, an alternative is to
use use
\href{http://hadoop.apache.org/docs/r0.19.0/hod_user_guide.html}{Hadoop
on Demand (HOD)}. In particular, HOD allows a MapReduce cluster to be
built upon an existing
\href{http://www.adaptivecomputing.com/products/open-source/torque/}{Torque
PBS job} cluster.

In general, Terrier can be configured to use an existing Hadoop
installation, by two changes:

\begin{enumerate}
\tightlist
\item
  Add the location of your \$HADOOP\_HOME/conf folder to the CLASSPATH
  environment variable before running Terrier.
\item
  Set property
  \texttt{terrier.plugins=org.terrier.utility.io.HadoopPlugin} in your
  terrier.properties file.
\item
  You must also ensure that there is a world-writable \texttt{/tmp}
  directory on Hadoop's default file system.
\end{enumerate}

This will allow Terrier to access the shared file system described in
your \texttt{core-site.xml}. If you also have the MapReduce job tracker
setup specified in \texttt{mapred-site.xml}, then Terrier can now
directly access the MapReduce job tracker to submit jobs.

\subsection{Using Hadoop On Demand
(HOD)}\label{using-hadoop-on-demand-hod}

If you don't have a dedicated Hadoop cluster yet, don't worry. Hadoop
provides a utility called Hadoop On Demand (HOD), which can use a
\href{http://www.adaptivecomputing.com/products/open-source/torque/}{Torque}
PBS cluster to create a Hadoop cluster. Terrier fully supports accessing
Hadoop clusters created by HOD, and can even call HOD to create the
cluster when its needed for a job. If your cluster is based on
\href{http://gridengine.sunsource.net/}{Sun Grid Engine}, this supports
Hadoop.

If you are using HOD, then Terrier can be configured to automatically
access it. Firstly, ensure HOD is working correctly, as described in the
HOD
\href{http://hadoop.apache.org/docs/r0.19.0/hod_user_guide.html}{user}
and
\href{http://hadoop.apache.org/docs/r0.19.0/hod_admin_guide.html}{admin}
guides. When Terrier wants to submit a MapReduce job, it will use the
\href{javadoc/org/terrier/utility/io/HadoopPlugin.html}{HadoopPlugin} to
request a MapReduce cluster from HOD. To configure this use the
following properties:

\begin{itemize}
\tightlist
\item
  \texttt{plugin.hadoop.hod} - set the full path to the local HOD
  executable. If this is not set, then HOD will not be used.
\item
  \texttt{plugin.hadoop.hod.nodes} - the number of nodes to request from
  HOD. Defaults to 6 nodes (sometimes CPUs).
\item
  \texttt{plugin.hadoop.hod.params} - any additional options you want to
  set on the HOD command line.
\end{itemize}

For more information on using HOD, see our
\href{javadoc/org/terrier/utility/io/HadoopPlugin.html}{HadoopPlugin
documentation}.

\subsection{Indexing with Hadoop
MapReduce}\label{indexing-with-hadoop-mapreduce}

We provide a \href{hadoop_indexing.html}{guide for configuring
single-pass indexing with MapReduce under Hadoop}.

\subsection{Developing MapReduce jobs with
Terrier}\label{developing-mapreduce-jobs-with-terrier}

Importantly, it should be possible to modify Terrier to perform other
information retrieval tasks using MapReduce. Terrier requires some
careful configuration to use in the MapReduce setting. The included,
\href{javadoc/org/terrier/utility/io/HadoopPlugin.html}{HadoopPlugin}
and
\href{javadoc/org/terrier/utility/io/HadoopUtility.html}{HadoopUtility}
should be used to link Terrier to Hadoop. In particular,
HadoopPlugin/HadoopUtility ensure that Terrier's share/ folder and the
terrier.properties file are copied to a shared space that all job tasks
can access. In the configure() method of the map and reduce tasks, you
must call \texttt{HadoopUtility.loadTerrierJob(jobConf)}. For more
information, see
\href{javadoc/org/terrier/utility/io/HadoopPlugin.html}{HadoopPlugin}.
Furthermore, we suggest that you browse the MapReduce indexing source
code, both for the map and reduce functions stored in the
\href{javadoc/org/terrier/structures/indexing/singlepass/hadoop/Hadoop_BasicSinglePassIndexer.html}{Hadoop\_BasicSinglePassIndexer}
and as well as the
\href{javadoc/org/terrier/structures/indexing/singlepass/hadoop/MultiFileCollectionInputFormat.html}{input
format} and
\href{javadoc/org/terrier/structures/indexing/singlepass/hadoop/SplitEmittedTerm.html}{partitioner}.

{[}\href{trec_examples.html}{Previous: TREC Experiment Examples}{]}
{[}\href{index.html}{Contents}{]} {[}\href{hadoop_indexing.html}{Next:
Hadoop MapReduce Indexing with Terrier}{]}

\begin{center}\rule{0.5\linewidth}{\linethickness}\end{center}

Webpage: \url{http://terrier.org}\\
Contact:
\href{mailto:terrier@dcs.gla.ac.uk}{\nolinkurl{terrier@dcs.gla.ac.uk}}\\
\href{http://www.dcs.gla.ac.uk/}{School of Computing Science}\\
Copyright (C) 2004-2015 \href{http://www.gla.ac.uk/}{University of
Glasgow}. All Rights Reserved.

~
