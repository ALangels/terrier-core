{[}\href{extend_retrieval.html}{Previous: Extending Retrieval}{]}
{[}\href{index.html}{Contents}{]} {[}\href{languages.html}{Next: Non
English language support}{]}\\[2\baselineskip]

\section{Pluggable Compression}\label{pluggable-compression}

\subsection{Introduction}\label{introduction}

The inverted index data structure contains a collection of postings
lists, a data structure which maintains information about the occurrence
of terms in documents. These are represented within Terrier as
implementations of two specific interfaces, namely
\href{javadoc/org/terrier/structures/postings/Posting.html}{Posting} and
\href{javadoc/org/terrier/structures/postings/IterablePosting.html}{IterablePosting}.
Terrier supports four different types of payload contained within each
Posting (or children interfaces):

\begin{itemize}
\tightlist
\item
  document identifier
\item
  term frequency
\item
  term frequency by field (e.g.: URL, title, body, incoming anchor text)
  a.k.a. field frequencies (implemented by
  \href{javadoc/org/terrier/structures/postings/FieldPosting.html}{FieldPosting})
\item
  term positions within the document (implemented by
  \href{javadoc/org/terrier/structures/postings/BlockPosting.html}{BlockPosting})
\end{itemize}

By default, Terrier compresses posting lists as a stream of postings. It
uses \href{http://en.wikipedia.org/wiki/Elias_gamma_coding}{Elias'
Gamma} compression schema (codec) to compress doc ids and term
positions; it uses
\href{http://en.wikipedia.org/wiki/Unary_coding}{Unary} codec to
compress term and field frequencies. The particular compression
configuration is defined by the CompressionConfiguration class. For more
information, please refer to
\href{javadoc/org/terrier/structures/bit/DirectInvertedOutputStream.html}{org.terrier.structures.bit.DirectInvertedOutputStream}
(and children) for documentation on postings compression, and
\href{javadoc/org/terrier/structures/postings/bit/BasicIterablePosting.html}{org.terrier.structures.postings.bit.BasicIterablePosting}
(and children) documentation for postings decompression. New in version
4.0, Terrier now supports more modern compression codecs, such as the
state-of-the-art PForDelta codec. In particular, a new integer
compression layer allows the transparent use of compression schemes from
\href{https://github.com/lemire/FastPFOR}{Java\_FastPFOR} by Daniel
Lemire, and
\href{http://data.linkedin.com/opensource/kamikaze}{Kamikaze} by
LinkedIn. Indeed, the new integer compression layer defines a new
CompressionConfiguration (namely
\href{javadoc/org/terrier/structures/integer/IntegerCodecCompressionConfiguration.html}{IntegerCodecCompressionConfiguration},
which can be configured to use various codecs for each compression
payload (document ids, term frequencies, field frequencies, term
positions):

\textbf{Name}

\textbf{Description}

\textbf{Codec Class name (in
\texttt{org.terrier.compression.integer.codec})}

VInt

\href{http://hadoop.apache.org}{Hadoop}'s Variable byte {[}1{]}
implementation

\href{javadoc/org/terrier/compression/integer/codec/VIntCodec.html}{VIntCodec}

Simple16

JavaFastPFOR's Simple16 {[}2,3{]} implementation

\href{javadoc/org/terrier/compression/integer/codec/LemireSimple16Codec.html}{LemireSimple16Codec}

Frame-of-Reference (FOR)

JavaFastPFOR's Frame-of-Reference {[}4{]} implementation

\href{javadoc/org/terrier/compression/integer/codec/LemireFORVBCodec.html}{LemireFORVBCodec}

NewPFD

JavaFastPFOR's NewPFD {[}5{]} implementation

\href{javadoc/org/terrier/compression/integer/codec/LemireNewPFDVBCodec.html}{LemireNewPFDVBCodec}

OptPFD

JavaFastPFOR's OptPFD {[}5{]} implementation

\href{javadoc/org/terrier/compression/integer/codec/LemireOptPFDVBCodec.html}{LemireOptPFDVBCodec}

FastPFOR

JavaFastPFOR's FastPFOR {[}6{]} implementation - NB: A larger chunk-size
is recommended for this codec.

\href{javadoc/org/terrier/compression/integer/codec/LemireFastPFORVBCodec.html}{LemireFastPFORVBCodec}

PForDelta

Linkedin's Kamikaze PForDelta {[}3,5{]}

\href{javadoc/org/terrier/compression/integer/codec/KamikazePForDeltaVBCodec.html}{KamikazePForDeltaVBCodec}

When using these codecs, the Terrier infrastructure (de)compresses
postings in chunks. The size of these chunks can be set at indexing time
using the properties
\texttt{index.inverted.compression.integer.chunk.size} for the direct
index, and \texttt{index.inverted.compression.integer.chunk.size} for
the inverted index.

\subsection{Indexing}\label{indexing}

Terrier 4.0 can perform classical two-pass indexing (i.e.
bin/trec\_terrier.sh -i), using the aforementioned codecs. To do so,
some properties have to be set. For instance, to store the direct and
inverted index compressed in blocks of 1024 posting using NewPFD codec:

\begin{verbatim}
index.direct.compression.integer.chunk.size=1024
indexing.direct.compression.configuration=org.terrier.structures.integer.IntegerCodecCompressionConfiguration
index.direct.compression.integer.chunk.size=1024
index.direct.compression.integer.ids.codec=LemireNewPFDVBCodec
index.direct.compression.integer.tfs.codec=LemireNewPFDVBCodec
indexing.inverted.compression.configuration=org.terrier.structures.integer.IntegerCodecCompressionConfiguration
index.inverted.compression.integer.chunk.size=1024
index.inverted.compression.integer.ids.codec=LemireNewPFDVBCodec
index.inverted.compression.integer.tfs.codec=LemireNewPFDVBCodec
index.inverted.compression.integer.fields.codec=LemireNewPFDVBCodec
index.inverted.compression.integer.blocks.codec=LemireNewPFDVBCodec
\end{verbatim}

You can also plug into Terrier a new compression schema by implementing
your own CompressionConfiguration. If IntegerCodec meets your
requirements, you can implement it, and directly use
IntegerCodecCompressionConfiguration. List of properties for indexing:

\textbf{Name}

\textbf{Description}

\textbf{Values}

\texttt{indexing.inverted.compression.configuration}\\
\texttt{indexing.direct.compression.configuration}

The class that defines the compression configuration to be used on the
inverted (direct) index at indexing time. Only classical indexing
supports pluggable compression.

org.terrier.structures.indexing.CompressionFactory\$BitCompressionConfiguration
(default);
org.terrier.structures.integer.IntegerCodecCompressionConfiguration

\texttt{index.inverted.compression.integer.chunk.size}\\
\texttt{index.direct.compression.integer.chunk.size}

Number of postings to be compressed at a time (used only w/
IntegerCodecCompressionConfiguration)

integer (default: 1024)

\texttt{index.inverted.compression.integer.ids.codec}\\
\texttt{index.direct.compression.integer.ids.codec}

The codec to be used to compress document identifiers in the inverted
index (used only w/ IntegerCodecCompressionConfiguration). For the
direct index, the codec to be used for the term identifiers.

See codecs table

\texttt{index.inverted.compression.integer.tfs.codec}\\
\texttt{index.direct.compression.integer.tfs.codec}

The codec to be used to compress term frequencies in the inverted
(direct) index (used only w/ IntegerCodecCompressionConfiguration)

"

\texttt{index.inverted.compression.integer.fields.codec}\\
\texttt{index.direct.compression.integer.fields.codec}

The codec to be used to compress field frequencies in the inverted
(direct) index (used only w/ IntegerCodecCompressionConfiguration,
optional)

"

\texttt{index.inverted.compression.integer.blocks.codec}\\
\texttt{index.direct.compression.integer.blocks.codec}

The codec to be used to compress term positions in the inverted (direct)
index (used only w/ IntegerCodecCompressionConfiguration, optional)

"

\subsection{Recompression}\label{recompression}

Inverted indices built by single-pass (i.e. bin/trec\_terrier.sh -i -j)
or MapReduce (i.e. bin/trec\_terrier.sh -i -H) indexing can be
re-compressed using the InvertedIndexRecompresser class. For example,
one can re-compress an inverted index using the OptPFD codec. This can
be performed using InvertedIndexRecompresser with the following
properties:

\begin{verbatim}
indexing.tmp-inverted.compression.configuration=org.terrier.structures.integer.IntegerCodecCompressionConfiguration
index.tmp-inverted.compression.integer.chunk.size=1024
index.tmp-inverted.compression.integer.ids.codec=LemireOptPFDVBCodec
index.tmp-inverted.compression.integer.tfs.codec=LemireOptPFDVBCodec
index.tmp-inverted.compression.integer.fields.codec=LemireOptPFDVBCodec
index.tmp-inverted.compression.integer.blocks.codec=LemireOptPFDVBCodec
\end{verbatim}

Please notice that InvertedIndexRecompresser overwrites the original
inverted index with the re-compressed one. Be sure to have one backup
copy of the inverted index before using InvertedIndexRecompresser.
Different codecs have different effects on index size and query response
time. When storage space is a concern, it is suggested to use Terrier's
default compression configuration (Simple16 and OptPFD are options too).
Instead, when the inverted index can fit in main memory, the best
practices derived in {[}7{]} recommend to use the FOR codec to reduce
the query response time, as follows:

\begin{verbatim}
index.direct.compression.integer.chunk.size=1024
indexing.direct.compression.configuration=org.terrier.structures.integer.IntegerCodecCompressionConfiguration
compression.direct.integer.ids.codec=LemireFORVBCodec
compression.direct.integer.tfs.codec=LemireFORVBCodec
indexing.inverted.compression.configuration=org.terrier.structures.integer.IntegerCodecCompressionConfiguration
compression.inverted.integer.ids.codec=LemireFORVBCodec
compression.inverted.integer.tfs.codec=LemireFORVBCodec
compression.inverted.integer.fields.codec=LemireFORVBCodec
compression.inverted.integer.blocks.codec=LemireFORVBCodec
\end{verbatim}

\subsection{Notes}\label{notes}

\begin{itemize}
\tightlist
\item
  The format of recompressed indices containing blocks was amended
  between 4.0 and 4.1. Terrier produces a warning if an index from 4.0
  with blocks enabled is opened. We recommend that you
  recompress/reindex as appropriate if this affects you. For more
  information, see
  \href{http://terrier.org/issues/browse/TR-320}{TR-320}.
\end{itemize}

\subsection{Citation Policy}\label{citation-policy}

If you make use of this package for efficient research, please cite:
Catena, M., Macdonald, C., Ounis, I.: On Inverted Index Compression for
Search Engine Efficiency. In: Proceedings of ECIR 2014.
{[}\href{http://www.dcs.gla.ac.uk/~craigm/publications/catena14compression.pdf}{PDF}{]}

\subsection{References}\label{references}

\begin{enumerate}
\tightlist
\item
  Williams, H.E., Zobel, J.: Compressing integers for fast file access.
  The Computer Journal 42 (1999)
\item
  Anh, V.N., Moffat, A.: Inverted index compression using word-aligned
  binary codes. Inf.Retr. 8 (1) (2005)
\item
  Zhang, J., Long, X., Suel, T.: Performance of compressed inverted list
  caching in search engines. In: Proc. WWW '08. (2008)
\item
  Goldstein, J., Ramakrishnan, R., Shaft, U.: Compressing relations and
  indexes. In: Proc. ICDE '98. (1998)
\item
  Yan, H., Ding, S., Suel, T.: Inverted index compression and query
  processing with optimized document ordering. In: Proc. WWW '09. (2009)
\item
  Lemire, D., Boytsov, L.: Decoding billions of integers per second
  through vectorization. Software: Practice and Experience (2013)
\item
  Catena, M., Macdonald, C., Ounis, I.: On Inverted Index Compression
  for Search Engine Efficiency. In: Proc. ECIR '14 (2014)
  {[}\href{http://www.dcs.gla.ac.uk/~craigm/publications/catena14compression.pdf}{PDF}{]}
\item
  Elias, P.: Universal codeword sets and representations of the
  integers. Trans. Info. Theory 21 (2) (1975)
\item
  Zukowski, M., Heman, S., Nes, N., Boncz, P.: Super-scalar RAM-CPU
  cache compression. In: Proc. ICDE '06. (2006)
\end{enumerate}

{[}\href{extend_retrieval.html}{Previous: Extending Retrieval}{]}
{[}\href{index.html}{Contents}{]} {[}\href{languages.html}{Next: Non
English language support}{]}

\begin{center}\rule{0.5\linewidth}{\linethickness}\end{center}

Webpage: \url{http://terrier.org}\\
Contact:
\href{mailto:terrier@dcs.gla.ac.uk}{\nolinkurl{terrier@dcs.gla.ac.uk}}\\
\href{http://www.dcs.gla.ac.uk/}{School of Computing Science}\\
Copyright (C) 2004-2015 \href{http://www.gla.ac.uk/}{University of
Glasgow}. All Rights Reserved.

~
