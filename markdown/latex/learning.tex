{[}\href{configure_retrieval.html}{Previous: Configure Retrieval}{]}
{[}\href{index.html}{Contents}{]} {[}\href{querylanguage.html}{Next:
Query Language}{]}\\[2\baselineskip]

\section{Learning to Rank with
Terrier}\label{learning-to-rank-with-terrier}

Since version 4.0, Terrier supports the deployment of many retrieval
features, and integration with learning to rank techniques. This page
explains how to configure Terrier to enable learning and application of
a learned model, while a worked example using the TREC .GOV corpus is
provided below.

\subsection{Introduction}\label{introduction}

Learning to rank is the ability to (a) use multiple features in a
uniform way during ranking, and (b) to learn an appropriate method to
combine those features. For learning, Terrier has the ability to
calculate multiple features, be they query-dependent (c.f. multiple
weighting models, such as BM25, PL2 etc), or be they query-independent
(c.f. Document Length).

If you wish to use Terrier with a learning-to-rank technique, you must
generate a LETOR-formatted file -- which can be done using Terrier's
Normalised2LETOROutputFormat class -- to provide as to a
learning-to-rank technique such as Jforests. A LETOR formatted file
looks as follows:

\begin{verbatim}
#1: featureName
#2: featureName
0 qid:1 1:2.9 2:9.4 # docid=clueweb09-00-01492
\end{verbatim}

At the top is an optional comment header giving the names of the
features. Then, each line has a `label', i.e. derived from relevance
assessments such as TREC qrel files, the query Id, and
featureId-featureValue pairs. Finally, each line can have the docno.

\subsection{Fat Component}\label{fat-component}

What is ``Fat'' about? Fat is a method of allowing many features to be
computed within one run of Terrier. In particular, computing every
feature for every posting of every query term is very expensive, and in
practice, unnecessary. (Indeed learning to rank becomes slower and less
accurate as the number of documents per query increases). Instead, Liu
{[}1{]} suggest ranking a \emph{sample} of documents using a simple
weighting model (e.g. BM25) is sufficient, before computing the features
on the documents in the sample.

However, once the sample has been identified, the posting lists have
been iterated, and it is no longer possible to compute the other
weighting model features. The Fat component {[}2{]} addresses this
problem, by storing copies of the postings for every document that makes
the top k retrieved documents. These can then be later used to calculate
other features.

\textbf{NB:} If you use the Fat component in your research, you should
cite {[}2{]}.

Moreover, in addition to the deployed features, the number of documents
to retrieve in the sample is a key parameter - for a study of this
parameter in Web search, see {[}3{]}.

In the following, we firstly define the classes related to Fat within
the learning component of Terrier, before showing how they can be
combined in pipelines for various applications.

\subsection{Fat Classes}\label{fat-classes}

\begin{itemize}
\item
  daat.FatFull: A DAAT exhaustive matching strategy based on daat.Full,
  however the postings for each document that enters the
  CandidateResultSet has its postings stored. In particular, the
  Posting.asWritablePosting() method is used to obtain a copy of a given
  posting, ``breaking it out'' of it's IterablePosting iterator. Returns
  a FatResultSet.
\item
  FatResultSet: A ResultSet which is fat, i.e. it has copies of all
  necessary statistics to compute other weighting models on the
  documents it contains. These statistics include the lengths of
  documents, the EntryStatistics of each term, the CollectionStatistics
  of the index, etc. There are various implementations:
  FatQueryResultSet, and FatCandidateResultSet. All FatResultSets are
  Writable, and hence can be serialized to disk for later use.
\item
  FatScoringMatching: Takes a FatResultSet obtained from a parent
  Matching class, and computes new scores based on a pre-determined
  weighting model.
\item
  FatRescoringMatching: Takes a FatResultSet obtained from a parent
  Matching class and re-scores the documents contained within it based
  upon a predetermined weighting model. This differs from
  FatScoringMatching in that it returns the original FatResultSet rather
  than a new QueryResultSet.
\item
  FatFeaturedScoringMatching: Permits many features to be calculated
  using a FatResultSet. In particular, takes a FatResultSet, and returns
  a FeaturedResultSet, which has each of a predefined number of
  features. Features can be one of three types:

  \begin{enumerate}
  \tightlist
  \item
    a query-dependent weighting model (denoted by a WMODEL prefix, and
    actually computed using FatScoringMatching).
  \item
    a query-independent weighting model (typically a feature) loaded by
    \href{javadoc/org/terrier/matching/models/StaticFeature.html}{StaticFeature}
    (NB: Terrier supports the loading of query independent features from
    a
    \href{javadoc/org/terrier/matching/models/StaticScoreModifierWeightingModel.html}{variety
    of input file formats}, however no methods of generating such
    features are provided out-of-the-box.) and
  \item
    the scores from a document score modifier.
  \end{enumerate}

  The names of features can be specified on a property, or read from a
  file. E.g.

\begin{verbatim}
WMODEL:BM25
WMODEL:PL2
QI:StaticFeature(OIS,/home/terrier4.0/var/results/data.inlinks.oos.gz)
DSM:org.terrier.matching.dsms.DFRDependenceScoreModifier
\end{verbatim}
\item
  WritableOutputFormat and FatResultsMatching: these classes permit
  FatResultSets to be written to file and read back in again, for the
  purposes of experimentation.
\end{itemize}

\subsection{Example Using .GOV corpus - Named Page
Retrieval}\label{example-using-.gov-corpus---named-page-retrieval}

In the following, we give an example of effective retrieval using
learning to rank, using the topics and qrels from the named page tasks
of the TREC 2002-2004 Web tracks. Other possible invocations of the Fat
framework are \protect\hyperlink{other}{discussed below}. Firstly, we
setup the Terrier environment. We use simple variables in the form of a
Unix Bash shell script, but this could be easily ported to a Windows
environment.

\begin{verbatim}
#available from http://ir.dcs.gla.ac.uk/test_collections/access_to_data.html 
CORPUS=/extra/Collections/TREC/GOV/

#available from http://trec.nist.gov/data/webmain.html
TR_TOPICS=/extra/TopicsQrels/TREC/GOV/namedpage/TREC2002/webnamed_page_topics.1-150.txt
VA_TOPICS=/extra/TopicsQrels/TREC/GOV/namedpage/TREC2003/topics.NP151-NP450.np
TE_TOPICS=/extra/TopicsQrels/TREC/GOV/namedpage/TREC2004/topics.WT04-1-WT04-225.np

TR_QRELS=/extra/TopicsQrels/TREC/GOV/namedpage/TREC2002/qrels.named-page.txt
VA_QRELS=/extra/TopicsQrels/TREC/GOV/namedpage/TREC2003/qrels.NP151-NP450.np
TE_QRELS=/extra/TopicsQrels/TREC/GOV/namedpage/TREC2004/qrels.WT04-1-WT04-225.np
\end{verbatim}

Firstly, we setup Terrier, and, if necessary, index the corpus using
blocks, and saving field information for the TITLE and body fields:

\begin{verbatim}
bin/trec_setup.sh $CORPUS

echo block.indexing=true >> etc/terrier.properties
echo FieldTags.process=TITLE,ELSE >> etc/terrier.properties

bin/trec_terrier.sh -i -j

Setting TERRIER_HOME to /home/terrier-4.0
INFO - TRECCollection read collection specification (4613 files)
INFO - Processing /extra/Collections/TREC/DOTGOV/G00/02.gz
INFO - Indexer using 2 fields
Starting building the inverted file (with blocks)...
...
INFO - Collection #0 took 3835 seconds to build the runs for 1247753 documents
INFO - Merging 2 runs...
INFO - Collection #0 took 151 seconds to merge
INFO - Collection #0 total time 3986
INFO - Optimising structure lexicon
INFO - lexicon has 2 fields
INFO - Optimising lexicon with 2759934 entries
INFO - All ids for structure lexicon are aligned, skipping .fsomapid file
Finished building the inverted index...
Time elapsed for inverted file: 3989.925
Time elapsed: 3990.343 seconds.
\end{verbatim}

Next, we wish to configure retrieval. We will use the Fat framework to
retrieve 1000 documents using the DPH weighting model, and then
calculate several additional query dependent and query independent
features:

\begin{verbatim}
#configure the features list file
cat <<EOF > etc/features.list
WMODEL:SingleFieldModel(BM25,0)
WMODEL:SingleFieldModel(BM25,1)
QI:SingleFieldModel(Dl,0)
QI:SingleFieldModel(Dl,1)
DSM:org.terrier.matching.dsms.DFRDependenceScoreModifier
DSM:org.terrier.matching.dsms.MRFDependenceScoreModifier
EOF
\end{verbatim}

Next, we want to retrieve results for the training topics. In this,
we're going to be calculating results with multiple features, as listed
in the \texttt{etc/features.list} file, so we use a series of Matching
classes: \href{javadoc/org/terrier/matching/daat/FatFull.html}{FatFull}
to make a
\href{javadoc/org/terrier/matching/FatResultSet.html}{FatResultSet}
(i.e. a ResultSet with extra posting information), and
\href{javadoc/org/terrier/matching/FatFeaturedScoringMatching.html}{FatFeaturedScoringMatching}
to add the additional features, and return a FeaturedResultSet. We then
add the document label from the qrels using LabelDecorator, and write
the results in a LETOR-compatible results file using
Normalised2LETOROutputFormat:

\begin{verbatim}
bin/trec_terrier.sh -r -Dtrec.model=DPH -Dtrec.topics=$TR_TOPICS -Dtrec.matching=FatFeaturedScoringMatching,org.terrier.matching.daat.FatFull -Dfat.featured.scoring.matching.features=FILE -Dfat.featured.scoring.matching.features.file=$PWD/etc/features.list  -Dtrec.querying.outputformat=Normalised2LETOROutputFormat -Dquerying.postprocesses.order=QueryExpansion,org.terrier.learning.LabelDecorator -Dquerying.postprocesses.controls=labels:org.terrier.learning.LabelDecorator,qe:QueryExpansion -Dquerying.default.controls=labels:on -Dlearning.labels.file=$TR_QRELS -Dtrec.results.file=tr.letor -Dproximity.dependency.type=SD


Setting TERRIER_HOME to /home/terrier-4.0
INFO - Structure meta reading lookup file into memory
INFO - Structure meta loading data file into memory
INFO - time to intialise index : 1.113
INFO - NP1 : america s century farms
INFO - Processing query: NP1: 'america s century farms'
INFO - Query NP1 with 3 terms has 3 posting lists
term america ks=1.0 es=term339011 Nt=94640 TF=229637 @{0 500765856 0} TFf=3801,225836
term centuri ks=1.0 es=term598109 Nt=39494 TF=81231 @{0 662970754 0} TFf=699,80532
term farm ks=1.0 es=term983771 Nt=40908 TF=178987 @{0 1009035982 2} TFf=2488,176499
Term: america qtw=1.0 es=term339011 Nt=94640 TF=229637 @{0 500765856 0} TFf=3801,225836
Term: centuri qtw=1.0 es=term598109 Nt=39494 TF=81231 @{0 662970754 0} TFf=699,80532
Term: farm qtw=1.0 es=term983771 Nt=40908 TF=178987 @{0 1009035982 2} TFf=2488,176499
INFO - Rescoring found 222 docs with +ve score using SingleFieldModel(BM25,0)
Term: america qtw=1.0 es=term339011 Nt=94640 TF=229637 @{0 500765856 0} TFf=3801,225836
Term: centuri qtw=1.0 es=term598109 Nt=39494 TF=81231 @{0 662970754 0} TFf=699,80532
Term: farm qtw=1.0 es=term983771 Nt=40908 TF=178987 @{0 1009035982 2} TFf=2488,176499
INFO - Rescoring found 1000 docs with +ve score using SingleFieldModel(BM25,1)
ngramC=1.0
read: term0 Nt=94640 TF=229637 @{0 500765856 0} TFf=3801,225836 => 0
read: term1 Nt=39494 TF=81231 @{0 662970754 0} TFf=699,80532 => 1
read: term2 Nt=40908 TF=178987 @{0 1009035982 2} TFf=2488,176499 => 2
INFO - Query NP1 with 3 terms has 3 posting lists
phrase term: america
phrase term: centuri
phrase term: farm
DFRDependenceScoreModifier altered scores for 1000 documents
read: term0 Nt=94640 TF=229637 @{0 500765856 0} TFf=3801,225836 => 0
read: term1 Nt=39494 TF=81231 @{0 662970754 0} TFf=699,80532 => 1
read: term2 Nt=40908 TF=178987 @{0 1009035982 2} TFf=2488,176499 => 2
INFO - Query NP1 with 3 terms has 3 posting lists
phrase term: america
phrase term: centuri
phrase term: farm
MRFDependenceScoreModifier altered scores for 1000 documents
INFO - Applying LabelDecorator
INFO - Writing results to /home/terrier-4.0/var/results/tr.letor
...
INFO - Settings of Terrier written to /home/terrier-4.0/var/results/tr.letor.settings
INFO - Finished topics, executed 150 queries in 361.379 seconds, results written to /home/terrier-4.0/var/results/tr.letor
\end{verbatim}

Lets a have a look at what was output into \texttt{tr.letor}:

\begin{verbatim}
# 1:score
# 2:WMODEL:SingleFieldModel(BM25,0)
# 3:WMODEL:SingleFieldModel(BM25,1)
# 4:QI:SingleFieldModel(Dl,0)
# 5:QI:SingleFieldModel(Dl,1)
# 6:DSM:org.terrier.matching.dsms.DFRDependenceScoreModifier
# 7:DSM:org.terrier.matching.dsms.MRFDependenceScoreModifier
1 qid:NP1 1:1.0 2:0.8508957593838526 3:1.0 4:0.38888888888888884 5:0.03218193520703712 6:0.9170723523167136 7:0.9586800382580136 #docid = 1241 docno = G00-65-2264297
1 qid:NP1 1:0.9020171593497216 2:0.8508957593838526 3:0.9169226938397537 4:0.38888888888888884 5:0.022634627762282773 6:1.0 7:1.0 #docid = 11394 docno = G00-04-3805407
0 qid:NP1 1:0.6440567566141826 2:0.0 3:0.85986678612829 4:0.0 5:0.22902810555674746 6:0.21370705023195802 7:0.3992684087689166 #docid = 921940 docno = G34-15-0261249
\end{verbatim}

The header reports the name of the features. ``score'' means the model
used to generate the sample, in our case DPH. After the header, for each
retrieved document for each query, there is a single line in the output.
The label obtained from the qrels file is the first entry on each row.

We repeat the retrieval step for the validation queries, this time from
the 2003 TREC task:

\begin{verbatim}
bin/trec_terrier.sh -r -Dtrec.model=DPH -Dtrec.topics=$VA_TOPICS -Dtrec.matching=FatFeaturedScoringMatching,org.terrier.matching.daat.FatFull -Dfat.featured.scoring.matching.features=FILE -Dfat.featured.scoring.matching.features.file=$PWD/etc/features.list  -Dtrec.querying.outputformat=Normalised2LETOROutputFormat -Dquerying.postprocesses.order=QueryExpansion,org.terrier.learning.LabelDecorator -Dquerying.postprocesses.controls=labels:org.terrier.learning.LabelDecorator,qe:QueryExpansion -Dquerying.default.controls=labels:on -Dlearning.labels.file=$VA_QRELS -Dtrec.results.file=va.letor -Dproximity.dependency.type=SD
\end{verbatim}

To obtain a learned model, we use the
\href{https://code.google.com/p/jforests/}{Jforests learning to rank
technique}, which is included with Terrier 4.0. In particular, we use
Jforests data preparation command to prepare the LETOR formatted results
files, then learn a LambdaMART learned model. The Jforests configuration
comes entirely from the Jforests website:

\begin{verbatim}
bin/anyclass.sh edu.uci.jforests.applications.Runner  --config-file etc/jforests.properties --cmd=generate-bin --ranking --folder var/results/ --file tr.letor  --file va.letor

#configure the jforests file
cat <<EOF > etc/jforests.properties
trees.num-leaves=7
trees.min-instance-percentage-per-leaf=0.25
boosting.learning-rate=0.05
boosting.sub-sampling=0.3
trees.feature-sampling=0.3
boosting.num-trees=2000
learning.algorithm=LambdaMART-RegressionTree
learning.evaluation-metric=NDCG
params.print-intermediate-valid-measurements=true
EOF

bin/anyclass.sh edu.uci.jforests.applications.Runner  --config-file etc/jforests.properties --cmd=train --ranking --folder var/results/ --train-file var/results/tr.bin --validation-file var/results/va.bin --output-model ensemble.txt
\end{verbatim}

Once the learned model (from Jforests, this is an XML file which takes
the form of a gradient boosted regression tree) is obtained in
\texttt{ensemble.txt}, we can use this to apply the learned model. The
configuration for Terrier is similar to retrieval for the training
topics, but we additionally use
\href{javadoc/org/terrier/matching/JforestsModelMatching.html}{JforestsModelMatching}
for application of the learned model, and output the final results using
the default, trec\_eval compatible
\href{javadoc/org/terrier/structures/outputformat/TRECDocnoOutputFormat.html}{TRECDocnoOutputFormat}:

\begin{verbatim}
bin/trec_terrier.sh -r -Dtrec.model=DPH -Dtrec.topics=$TE_TOPICS -Dtrec.matching=JforestsModelMatching,FatFeaturedScoringMatching,org.terrier.matching.daat.FatFull -Dfat.featured.scoring.matching.features=FILE -Dfat.featured.scoring.matching.features.file=$PWD/etc/features.list -Dtrec.results.file=te.res -Dfat.matching.learned.jforest.model=$PWD/ensemble.txt -Dfat.matching.learned.jforest.statistics=$PWD/var/results/jforests-feature-stats.txt -Dproximity.dependency.type=SD
\end{verbatim}

Finally, for comparison, we additionally make a simple DPH run:

\begin{verbatim}
bin/trec_terrier.sh -r -Dtrec.model=DPH -Dtrec.topics=$TE_TOPICS
\end{verbatim}

On evaluating the two runs using trec\_eval for Mean Reciprocal Rank, we
find a marked increase in effectiveness, despite the deployment of no
Web-specific features (such as anchor text, URL or link analysis
features).

\begin{verbatim}
trec_eval -m recip_rank $TE_QRELS var/results/DPH_0.res
recip_rank              all 0.4447
trec_eval  -m recip_rank $TE_QRELS var/results/te.res 
recip_rank              all 0.5201
\end{verbatim}

\textbf{NB:} You can equally use Terrier's
\href{evaluation.html}{evaluation tool} to conduct the evaluation.

\href{}{}

\subsection{Other possible usages}\label{other-possible-usages}

In the following, we give typical configurations for using the
learning/fat components of Terrier.

\subsubsection{From inverted index -\textgreater{} LETOR file with many
features}\label{from-inverted-index---letor-file-with-many-features}

\begin{verbatim}
bin/trec_terrier.sh -r -Dtrec.matching=FatFeaturedScoringMatching,org.terrier.matching.daat.FatFull -Dfat.featured.scoring.matching.features=FILE -Dfat.featured.scoring.matching.features.file=/path/to/list.features -Dtrec.querying.outputformat=org.terrier.learning.Normalised2LETOROutputFormat
\end{verbatim}

\subsubsection{From inverted index -\textgreater{} Fat result file
-\textgreater{} LETOR file with many
features}\label{from-inverted-index---fat-result-file---letor-file-with-many-features}

You can save intermediate FatResultSets, to that can go back and compute
different sets of features without retrieval from the inverted index.

\begin{verbatim}
bin/trec_terrier.sh -r -Dtrec.matching=org.terrier.matching.daat.FatFull -Dtrec.querying.outputformat=org.terrier.applications.WritableOutputFormat

bin/trec_terrier.sh -r -Dtrec.matching=FatFeaturedScoringMatching,FatResultsMatching -Dfat.results.matching.file=bla.fat.res.gz  -Dfat.featured.scoring.matching.features=FILE -Dfat.featured.scoring.matching.features.file=/path/to/list.features -Dtrec.querying.outputformat=org.terrier.learning.Normalised2LETOROutputFormat
\end{verbatim}

\subsubsection{From inverted index -\textgreater{} Final Ranking having
applied learned model to
documents}\label{from-inverted-index---final-ranking-having-applied-learned-model-to-documents}

\begin{verbatim}
bin/trec_terrier.sh -r -Dtrec.matching=JforestsModelMatching,FatFeaturedScoringMatching,org.terrier.matching.daat.FatFull -Dfat.featured.scoring.matching.features=FILE -Dfat.featured.scoring.matching.features=$PWD/list.features -Dfat.matching.learned.jforest.model=/path/to/jforest.model
\end{verbatim}

\subsection{References}\label{references}

\begin{enumerate}
\tightlist
\item
  Tie-Yan Lui. Learning to Rank for Information Retrieval. Foundations
  \& Trends in Information Retrieval. 3(3):225-331, 2009.
\item
  Craig Macdonald, Rodrygo L.T. Santos, Iadh Ounis and Ben He. About
  Learning Models with Multiple Query Dependent Features. Transactions
  on Information Systems. 31(3):1-39. 2013.
\item
  Craig Macdonald, Rodrygo Santos and Iadh Ounis. The Whens and Hows of
  Learning to Rank. Information Retrieval 16(5):584-628. 2012.
\end{enumerate}

{[}\href{configure_retrieval.html}{Previous: Configure Retrieval}{]}
{[}\href{index.html}{Contents}{]} {[}\href{querylanguage.html}{Next:
Query Language}{]}

\begin{center}\rule{0.5\linewidth}{\linethickness}\end{center}

Webpage: \url{http://terrier.org}\\
Contact:
\href{mailto:terrier@dcs.gla.ac.uk}{\nolinkurl{terrier@dcs.gla.ac.uk}}\\
\href{http://www.dcs.gla.ac.uk/}{School of Computing Science}\\
Copyright (C) 2004-2015 \href{http://www.gla.ac.uk/}{University of
Glasgow}. All Rights Reserved.

~
