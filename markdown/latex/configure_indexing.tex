{[}\href{configure_general.html}{Previous: Configuring Terrier}{]}
{[}\href{index.html}{Contents}{]}
{[}\href{configure_retrieval.html}{Next: Configuring Retrieval}{]}

\section{Configuring Indexing in
Terrier}\label{configuring-indexing-in-terrier}

\subsection{Indexing Overview}\label{indexing-overview}

The indexing process in Terrier is described
\href{http://terrier.org/docs/current/basicComponents.html}{here}

Firstly, a
\href{javadoc/org/terrier/indexing/TRECCollection.html}{Collection}
object extracts the raw content of each individual document (from a
collection of documents) and hands it in to a
\href{javadoc/org/terrier/indexing/Document.html}{Document} object. The
Document object then removes any unwanted content (e.g., from a
particular document tag) and gives the resulting text to a
\href{javadoc/org/terrier/indexing/tokenisation/Tokeniser.html}{Tokeniser}
object. Finally, the tokeniser object converts the text into a stream of
tokens that represent the content of the document

.

By default, Terrier uses
\href{javadoc/org/terrier/indexing/TRECCollection.html}{TRECCollection},
which parses corpora in TREC format. In particular, in TREC-formatted
files, there are many documents delimited by
\texttt{\textless{}DOC\textgreater{}\textless{}/DOC\textgreater{}} tags,
as in the following example:

\begin{verbatim}
<DOC>
<DOCNO> doc1 </DOCNO>
Content of the document does here
</DOC>
<DOC>
...
\end{verbatim}

For corpora in other formats, you will need to change the Collection
object being used, by setting the property
\texttt{trec.collection.class}. Here are some options:

\begin{itemize}
\tightlist
\item
  \href{javadoc/org/terrier/indexing/TRECCollection.html}{TRECCollection}
  - Parses TREC formatted corpora, delimited by the
  \texttt{\textless{}DOC\textgreater{}\textless{}/DOC\textgreater{}}
  tags.
\item
  \href{javadoc/org/terrier/indexing/TRECWebCollection.html}{TRECWebCollection}
  - As TRECCollection, but additionally parses DOCHDR tags, which
  contain the URL of each document. TREC Web and Blog corpora such as
  WT2G, WT10G, .GOV, .GOV2, Blogs06 and Blogs08 are supported.
\item
  \href{javadoc/org/terrier/indexing/WARC09Collection.html}{WARC09Collection}
  - Parses corpora in WARC version 0.9 format, such as UK-2006.
\item
  \href{javadoc/org/terrier/indexing/WARC018Collection.html}{WARC018Collection}
  - Parses corpora in WARC version 0.18 format, such as ClueWeb09.
\item
  \href{javadoc/org/terrier/indexing/SimpleFileCollection.html}{SimpleFileCollection}
  - Parses HTML, Microsoft Word/Excel/Powerpoint, PDF, text documents,
  etc., one document per file. For a guide on how to use this class, see
  the
  \href{http://ir.dcs.gla.ac.uk/wiki/Terrier/CollectionOfFiles}{collection
  of files guide} on the Terrier wiki.
\item
  \href{javadoc/org/terrier/indexing/SimpleXMLCollection.html}{SimpleXMLCollection}
  - Like TRECCollection, but where the input is valid XML.
\item
  \href{javadoc/org/terrier/indexing/SimpleMedlineXMLCollection.html}{SimpleMedlineXMLCollection}
  - Special version of SimpleXMLCollection for modern Medline documents.
\end{itemize}

Except for the special-purpose collections (SimpleFileCollection,
SimpleXMLCollection, and SimpleMedlineXMLCollection), all other
Collection implementations allow for different
\href{javadoc/org/terrier/indexing/Document.html}{Document}
implementations to be used, by specifying the
\texttt{trec.document.class} property. By default, these collections use
\href{javadoc/org/terrier/indexing/TaggedDocument.html}{TaggedDocument}.
The available Document implementations are:

\begin{itemize}
\tightlist
\item
  \href{javadoc/org/terrier/indexing/TaggedDocument.html}{TaggedDocument}
  - Models a tagged document (e.g., an HTML or TREC document). Note that
  from Terrier 4.0, this class replaced HTMLDocument and TRECDocument.
\item
  \href{javadoc/org/terrier/indexing/FileDocument.html}{FileDocument} -
  Models a document which corresponds to a single, plain-text file.
\item
  \href{javadoc/org/terrier/indexing/PDFDocument.html}{PDFDocument},
  \href{javadoc/org/terrier/indexing/MSExcelDocument.html}{MSExcelDocument},
  and
  \href{javadoc/org/terrier/indexing/MSWordDocument.html}{MSWordDocument}
  - Model PDF, MS Excel (.xls), MS Powerpoint (.ppt), and MS Word (.doc)
  documents, respectively.
\end{itemize}

Finally, all Document implementations can specify their own
\href{javadoc/org/terrier/indexing/tokenisation/Tokeniser.html}{Tokeniser}
implementation. By default, Terrier uses the
\href{javadoc/org/terrier/indexing/tokenisation/EnglishTokeniser.html}{EnglishTokeniser}.
When \href{languages.html}{indexing non-English corpora}, a different
Tokeniser implementation can be specified by the \texttt{tokeniser}
property.

\subsubsection{Basic indexing setup}\label{basic-indexing-setup}

For now, we'll stick to TRECCollection, which can be used for all TREC
corporas from Disks 1\&2 until Blogs08, including WT2G, .GOV, .GOV2,
etc. TRECCollection can be further configured.

\begin{itemize}
\tightlist
\item
  Set \texttt{TrecDocTags.doctag} to denote the marker tag for document
  boundaries (usually \texttt{DOC}).
\item
  \texttt{TrecDocTags.idtag} denotes the tag that contains the
  \texttt{DOCNO} of the document.
\item
  \texttt{TrecDocTags.skip} denotes tags that should not be parsed in
  this collection (for instance, the \texttt{DOCHDR} tags of TREC Web
  collections).
\end{itemize}

Note that the specified tags are case-sensitive, but this can be relaxed
by setting the \texttt{TrecDocTags.casesensitive} property to false.
Furthermore, TRECCollection also supports the addition of the contents
of tags to the meta index. This is useful if you wish to present these
during retrieval (e.g. the URL of the document, or the date). To use
this, the tags in the TREC collection file need to be in a fixed order,
beginning with the \texttt{DOC} and \texttt{DOCNO} tags, followed by the
tags to be added to the meta index and specified by the
\texttt{TrecDocTags.propertytags} property. Any tags occurring after the
property tags will be indexed as if they contain text (unless excluded
by \texttt{TrecDocTags.skip}). The name of the entries in the meta index
must be the same as the tag names. Moreover, as with any entries added
to the meta index, these entries must be specified in the
\texttt{indexer.meta.forward.keys} property and the maximum length of
each tag must be given in the \texttt{indexer.meta.forward.keylens}
property.

\href{}{}

Terrier has the ability to record the frequency with which terms occur
in various fields of documents. The required fields are specified by the
\texttt{FieldTags.process} property. For example, to note when a term
occurs in the TITLE or H1 HTML tags of a document, set
\texttt{FieldTags.process=TITLE,H1}. FieldTags are case-insensitive.
There is a special field called ELSE, which contains all terms not in
any other specified field.

The indexer iterates through the documents of the collection and sends
each term found through the
\href{javadoc/org/terrier/terms/TermPipeline.html}{TermPipeline}. The
TermPipeline transforms the terms, and can remove terms that should not
be indexed. The TermPipeline chain in use is
\texttt{termpipelines=Stopwords,PorterStemmer}, which removes terms from
the document using the
\href{javadoc/org/terrier/terms/Stopwords.html}{Stopwords} object, and
then applies Porter's Stemming algorithm for English to the terms
(\href{javadoc/org/terrier/terms/PorterStemmer.html}{PorterStemmer}). If
you want to use a different stemmer, this is the point at which it
should be called.

The term pipeline can also be configured at indexing time to skip
various tokens. Set a comma-delimited list of tokens to skip in the
property \texttt{termpipelines.skip}. The same property works at
retrieval time also.

The indexers are more complicated. Each class can be configured by
several properties. Many of these alter the memory usage of the classes.

\begin{itemize}
\tightlist
\item
  \texttt{indexing.max.tokens} - The maximum number of tokens the
  indexer will attempt to index in a document. If 0, then all tokens
  will be indexed (default).
\item
  \texttt{ignore.empty.documents} - whether to assign document Ids to
  empty documents. Defaults to true.
\item
  \texttt{indexing.max.docs.per.builder} - Maximum number of documents
  in an index before a new index is created and merged later.
\item
  \texttt{indexing.builder.boundary.docnos} - Comma-delimited list of
  docnos of documents that force the index being created to be
  completed, and a new index to be commenced. An alternative to
  \texttt{indexing.max.docs.per.builder}.
\end{itemize}

For the
\href{javadoc/org/terrier/structures/indexing/classical/BlockIndexer.html}{BlockIndexer}:

\begin{itemize}
\tightlist
\item
  \texttt{block.indexing} - Whether block indexing should be enabled.
  Defaults to false.
\item
  \texttt{blocks.size} - How many terms should be in one block. If you
  want to use phrasal search, this needs to be 1 (default).
\item
  \texttt{blocks.max} - Maximum number of blocks in a document. After
  this number of blocks, all subsequent terms will be in the same block.
  Defaults to 100,000.
\end{itemize}

Once terms have been processed through the TermPipeline, they are
aggregated by the
\href{javadoc/org/terrier/structures/indexing/DocumentPostingList.html}{DocumentPostingList}
and the
\href{javadoc/org/terrier/structures/indexing/LexiconMap.html}{LexiconMap}.
These have a few properties:

\begin{itemize}
\tightlist
\item
  \texttt{max.term.length} - Maximum length of one term, in characters.
\end{itemize}

Document metadata is recorded in a
\href{javadoc/org/terrier/structures/MetaIndex.html}{MetaIndex}
structure. For instance, such metadata could include the DOCNO and URL
of each document, which the system can use to represent the document
during retrieval. The MetaIndex can be configured to take note of
various document attributes during indexing. The available attributes
depend on those provided by the
\href{javadoc/org/terrier/indexing/Document.html}{Document}
implementation. MetaIndex can be configured using the following
properties:

\begin{itemize}
\tightlist
\item
  \texttt{indexer.meta.forward.keys} - Comma-delimited list of document
  attributes to store in the MetaIndex. e.g.
  \texttt{indexer.meta.forward.keys=docno} or
  \texttt{indexer.meta.forward.keys=url,title}. If this property is set
  the following property needs also to be set.
\item
  \texttt{indexer.meta.forward.keylens} - Comma-delimited list of the
  maximum length of the attributes to be stored in the MetaIndex.
  Defaults to 20. The number of key lengths here should be identical to
  the number keys in indexer.meta.forward.keys.
\item
  \texttt{indexer.meta.reverse.keys} - Comma-delimited list of document
  attributes that \emph{uniquely} denote a document. These mean that
  given a document attribute value, a single document can be identified.
\end{itemize}

Note that for presenting results to a user, additional indexing
configuration is required. See \href{terrier_http.html}{Web-based
Terrier} for more information.

\subsubsection{Classical two-pass
indexing}\label{classical-two-pass-indexing}

Terrier supports three types of indexing: ``classical two-pass'',
``single-pass'' and ``MapReduce''. All three methods create an identical
inverted index, that produces identical retrieval effectiveness.
However, they differ on other characteristics, namely their support for
query expansion, and the scalability and efficiency when indexing large
corpora. The choice of indexing method is likely to be driven by your
need for query expansion, and the scale of the data you are working
with. In particular, only classical twio-pass indexing directly creates
a direct index, which is used for query expansion. However, classical
two-pass indexing doesn't scale to large corpora (maximum practical is
about 25 million documents). Single pass indexing is faster, but doesn't
create a direct index. MapReduce indexing can be used when you have very
large data (e.g. 50+ million documents), and already have an existing
Hadoop cluster. If you do create an index that doesn't have a direct
index, you can create one later using

\begin{verbatim}
--inverted2direct
\end{verbatim}

flag when calling trec\_terrier.sh.

This subsection describes the classical indexing implemented by
BasicIndexer and BlockIndexer. For single-pass indexing, see the next
subsection.

The LexiconMap is flushed to disk every \texttt{bundle.size} documents.
If memory during indexing is a concern, then reduce this property to
less than its default 2500. However, more temporary lexicons will be
created. The rate at which the temporary lexicons are merged is
controlled by the \texttt{lexicon.builder.merge.lex.max} property,
though we have found 16 to be a good compromise.

Once all documents in the index have been created, the InvertedIndex is
created by the
\href{javadoc/org/terrier/structures/indexing/classical/InvertedIndexBuilder.html}{InvertedIndexBuilder}.
As the entire DirectIndex cannot be inverted in memory, the
InvertedIndexBuilder takes several iterations, selecting a few terms,
scanning the direct index for them, and then writing out their postings
to the inverted index. If it takes too many terms at once, Terrier can
run out of memory. Reduce the property
\texttt{invertedfile.processpointers} from its default 20,000,000 and
rerun (default is only 2,000,000 for block indexing, which is more
memory intensive). See the
\href{javadoc/org/terrier/structures/indexing/classical/InvertedIndexBuilder.html}{InvertedIndexBuilder}
for more information about the inversion and term selection strategies.

\subsubsection{Single-pass indexing}\label{single-pass-indexing}

Single-pass indexing is implemented by the classes
\href{javadoc/org/terrier/structures/indexing/singlepass/BasicSinglePassIndexer.html}{BasicSinglePassIndexer}
and
\href{javadoc/org/terrier/structures/indexing/singlepass/BasicSinglePassIndexer.html}{BlockSinglePassIndexer}.
Essentially, instead of building a direct file from the collection, term
posting lists are held in memory, and written to disk as `run' when
memory is exhausted. These are then merged to form the lexicon and the
inverted file. Note that no direct index is created - indeed, the
single-pass indexing is much faster than classical two-pass indexing
when the direct index is not required. If the direct index is required,
then this can be built from the inverted index using the
\href{javadoc/org/terrier/structures/indexing/singlepass/Inverted2DirectIndexBuilder.html}{Inverted2DirectIndexBuilder}.

The single-pass indexer can be used by using the \texttt{-i\ -j} command
line argument to TrecTerrier.

The majority of the properties configuring the single-pass indexer are
related to memory consumption, and how it decides that memory has been
exhausted. Firstly, the indexer will commit a run to disk when free
memory falls below the threshold set by \texttt{memory.reserved} (50MB).
To ensure that this doesn't happen too soon, 85\% of the possible heap
must be allocated (controlled by the property
\texttt{memory.heap.usage}). This check occurs every 20 documents
(\texttt{docs.checks}).

Single-pass indexing is significantly quicker than two-pass indexing.
However, there are some configuration points to be aware of. In
particular, it makes much use of the memory to reduce disk IO. For Java
6, we recommend adding the \texttt{\ -XX:-UseGCOverheadLimit} to the
command line. Moreover, for very large indices, many files have to be
opened during merging, possibly exhausting the maximum number of allowed
open files. Refer to your operating system documentation to increase
this limit.

Notably, single-pass indexing does not build a direct index. However, a
direct index can be build later using the \texttt{-id} command line
argument to TrecTerrier.

\subsubsection{MapReduce indexing}\label{mapreduce-indexing}

For large-scale collections, Terrier provides a MapReduce based indexing
system. For more details, please see \href{hadoop_indexing.html}{Hadoop
MapReduce Indexing with Terrier}.

\subsubsection{Real-time indexing}\label{real-time-indexing}

Terrier also supports the real-time indexing of document collections
using MemoryIndex and IncrementalIndex structures, allowing for new
documents to be added to the index at later points in time. For more
details, please see \href{realtime_indices.html}{Real-time Index
Structures}.

\subsection{Compression}\label{compression}

By default, Terrier uses Elias-Gamma and Elias-Unary algorithms for
ensuring a highly compressed direct and inverted indices, however
starting with version 4.0 Terrier now has support for a variety of
state-of-the-art compression schemes including PForDelta. For more
information about configuring the compression used for indexing, see the
\href{compression.html}{documentation on compression}.

\subsection{More about Block Indexing}\label{more-about-block-indexing}

\subsubsection{What are blocks?}\label{what-are-blocks}

A block is a unit of text in a document. When you index using blocks,
you tell Terrier to save positional information with each term.
Depending on how Terrier has been configured, a block can be of size 1
or larger. Size 1 means that the exact position of each term can be
determined. For size \textgreater{} 1, the block id is incremented after
every N terms. You can configure the size of a block using the property
\texttt{blocks.size}.

\subsubsection{How do I use blocks?}\label{how-do-i-use-blocks}

You can enable block indexing by setting the property
\texttt{block.indexing} to \texttt{true} in your terrier.properties
file. This ensures that the Indexer used for indexing is the
BlockIndexer, not the BasicIndexer (or BlockSinglePassIndexer instead of
BasicSinglePassIndexer). When loading an index, Terrier will detect that
the index has block information saved and use the appropriate classes
for reading the index files.

You can use the positional information when doing retrieval. For
instance, you can search for documents matching a phrase, e.g.
\texttt{"Terabyte\ retriever"}, or where the words occur near each
other, e.g. \texttt{"indexing\ blocks"\textasciitilde{}20}.

\subsubsection{What changes when I use block
indexing?}\label{what-changes-when-i-use-block-indexing}

When you enable the property \texttt{block.indexing}, the indexer used
is the BlockIndexer, not the BasicIndexer (if you're using single-pass
indexing, it is the BlockSinglePassIndexer, not the
BasicSinglePassIndexer that is used). The created DirectIndex and
InvertedIndex use a different format, which includes the blockids for
each posting, and can be read by BlockDirectIndex and
BlockInvertedIndex, respectively. During two-pass indexing,
BlockLexicons are created to keep track of how many blocks are in use
for a term. However, at the last stage of rewriting the lexicon at the
end of inverted indexing, the BlockLexicon is rewritten as a normal
Lexicon, as the block information can be guessed during retrieval.

{[}\href{configure_general.html}{Previous: Configuring Terrier}{]}
{[}\href{index.html}{Contents}{]}
{[}\href{configure_retrieval.html}{Next: Configuring Retrieval}{]}

\begin{center}\rule{0.5\linewidth}{\linethickness}\end{center}

Webpage: \url{http://terrier.org}\\
Contact:
\href{mailto:terrier@dcs.gla.ac.uk}{\nolinkurl{terrier@dcs.gla.ac.uk}}\\
\href{http://www.dcs.gla.ac.uk/}{School of Computing Science}\\
Copyright (C) 2004-2015 \href{http://www.gla.ac.uk/}{University of
Glasgow}. All Rights Reserved.
