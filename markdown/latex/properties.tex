{[}\href{hadoop_indexing.html}{Previous: Hadoop MapReduce Indexing with
Terrier}{]} {[}\href{index.html}{Contents}{]}
{[}\href{terrier_develop.html}{Next: Developing with Terrier}{]}\\

\section{Description of Configurable properties of
Terrier}\label{description-of-configurable-properties-of-terrier}

Terrier allows the user to configure many different aspects of the
framework, in order to be adaptable to the specific needs of different
applications. Here, we describe the properties that are used while
indexing or retrieving. A sample of how to set up the basic properties
can be found in
\href{../etc/terrier.properties.sample}{etc/terrier.properties.sample}.
This page contains many of the properties in Terrier, broken down by
category: \protect\hyperlink{general}{General},
\protect\hyperlink{indexing}{Indexing},
\protect\hyperlink{retrieval}{Retrieval},
\protect\hyperlink{desktop}{Desktop Search} and
\protect\hyperlink{misc}{Miscellaneous}.

\subsection{\texorpdfstring{\href{}{General
properties}}{General properties}}\label{general-properties}

\begin{longtable}[]{@{}ll@{}}
\toprule
Property & \textbf{terrier.setup}\tabularnewline
Used in &
\href{javadoc/org/terrier/utility/ApplicationSetup.html}{org.terrier.utility.ApplicationSetup}\tabularnewline
Possible values & Absolute directory path\tabularnewline
Default value & not specified\tabularnewline
Configures & Specifies where Terrier finds the terrier.properties file,
which is usually in the etc/ directory. Analogous to terrier.etc
property\tabularnewline
\bottomrule
\end{longtable}

\begin{longtable}[]{@{}ll@{}}
\toprule
Property & \textbf{terrier.home}\tabularnewline
Used in &
\href{javadoc/org/terrier/utility/ApplicationSetup.html}{org.terrier.utility.ApplicationSetup}\tabularnewline
Possible values & Absolute directory path\tabularnewline
Default value & not specified\tabularnewline
Configures & ApplicationSetup.TERRIER\_HOME. Where Terrier is
installed.\tabularnewline
\bottomrule
\end{longtable}

\begin{longtable}[]{@{}ll@{}}
\toprule
Property & \textbf{terrier.etc}\tabularnewline
Used in &
\href{javadoc/org/terrier/utility/ApplicationSetup.html}{org.terrier.utility.ApplicationSetup}\tabularnewline
Possible values & Absolute directory path\tabularnewline
Default value & TERRIER\_HOME + ``etc/''\tabularnewline
Configures & TERRIER\_ETC. Where terrier finds it's terrier.properties
file if -Dterrier.setup is not specified\tabularnewline
\bottomrule
\end{longtable}

\begin{longtable}[]{@{}ll@{}}
\toprule
Property & \textbf{terrier.share}\tabularnewline
Used in &
\href{javadoc/org/terrier/utility/ApplicationSetup.html}{org.terrier.utility.ApplicationSetup},
\href{javadoc/org/terrier/terms/Stopwords.html}{org.terrier.terms.Stopwords}\tabularnewline
Possible values & Absolute directory path\tabularnewline
Default value & TERRIER\_HOME + ``share/''\tabularnewline
Configures & ApplicationSetup.TERRIER\_SHARE. Where static distribution
files are found, for instance the stopword files.\tabularnewline
\bottomrule
\end{longtable}

Property

\textbf{terrier.var}

Used in

\href{javadoc/org/terrier/utility/ApplicationSetup.html}{org.terrier.utility.ApplicationSetup},
\href{javadoc/org/terrier/applications/desktop/filehandling/WindowsFileOpener.html}{org.terrier.applications.desktop.filehandling.WindowsFileOpener},
\href{javadoc/org/terrier/structures/Index.html}{org.terrier.structures.Index}

Possible values

Absolute directory path

Default value

TERRIER\_HOME + ``var/''

Configures

TERRIER\_VAR. Where Terrier puts files that it creates, e.g. indices and
results files.

\begin{longtable}[]{@{}ll@{}}
\toprule
Property & \textbf{terrier.plugins}\tabularnewline
Used in &
\href{javadoc/org/terrier/utility/ApplicationSetup.html}{org.terrier.utility.ApplicationSetup}\tabularnewline
Possible values & A comma-separated list of plugins.\tabularnewline
Default value & not specified\tabularnewline
Configures & The list of plugins to be preloaded.\tabularnewline
\bottomrule
\end{longtable}

\begin{longtable}[]{@{}ll@{}}
\toprule
Property & \textbf{log4j.config}\tabularnewline
Used in &
\href{javadoc/org/terrier/utility/ApplicationSetup.html}{org.terrier.utility.ApplicationSetup}\tabularnewline
Possible values & A valid log4j configuration file\tabularnewline
Default value & terrier-log.xml\tabularnewline
Configures & ApplicationSetup.LOG4J\_CONFIG. The configuration file used
by log4j.\tabularnewline
\bottomrule
\end{longtable}

\hypertarget{indexing}{\subsection{\texorpdfstring{\href{}{Indexing}}{Indexing}}\label{indexing}}

\begin{longtable}[]{@{}ll@{}}
\toprule
Property & \textbf{terrier.index.path}\tabularnewline
Used in &
\href{javadoc/org/terrier/utility/ApplicationSetup.html}{org.terrier.utility.ApplicationSetup},
\href{javadoc/org/terrier/indexing/SimpleFileCollection.html}{org.terrier.indexing.SimpleFileCollection},
\href{javadoc/org/terrier/indexing/TRECCollection.html}{org.terrier.indexing.TRECCollection}\tabularnewline
Possible values & fully path of a directory\tabularnewline
Default value & TERRIER\_VAR + ``index/''\tabularnewline
Configures & TERRIER\_INDEX\_PATH. The name of the directory in which
the data structures created by Terrier are stored\tabularnewline
\bottomrule
\end{longtable}

\begin{longtable}[]{@{}ll@{}}
\toprule
Property & \textbf{terrier.index.prefix}\tabularnewline
Used in &
\href{javadoc/org/terrier/utility/ApplicationSetup.html}{org.terrier.utility.ApplicationSetup},
\href{javadoc/org/terrier/structures/indexing/Indexer.html}{org.terrier.structures.indexing.Indexer}\tabularnewline
Possible values & Filename prefix for all the indices\tabularnewline
Default value & ``data''\tabularnewline
Configures & TERRIER\_INDEX\_PREFIX. Filename prefix for all the
indices.\tabularnewline
\bottomrule
\end{longtable}

\begin{longtable}[]{@{}ll@{}}
\toprule
Property & \textbf{stopwords.filename}\tabularnewline
Used in &
\href{javadoc/org/terrier/terms/Stopwords.html}{org.terrier.terms.Stopwords}\tabularnewline
Possible values & absolute path to file\tabularnewline
Default value & TERRIER\_SHARE + ``stopword-list.txt''\tabularnewline
Configures & The name of the file which contains a list of
stopwords.\tabularnewline
\bottomrule
\end{longtable}

\begin{longtable}[]{@{}ll@{}}
\toprule
Property & \textbf{collection.spec}\tabularnewline
Used in &
\href{javadoc/org/terrier/utility/ApplicationSetup.html}{org.terrier.utility.ApplicationSetup},
\href{javadoc/org/terrier/indexing/SimpleFileCollection.html}{org.terrier.indexing.SimpleFileCollection},
\href{javadoc/org/terrier/indexing/TRECCollection.html}{org.terrier.indexing.TRECCollection}\tabularnewline
Possible values & Absolute filename\tabularnewline
Default value & TERRIER\_ETC + value of
``collection.spec''\tabularnewline
Configures & COLLECTION\_SPEC. Where the indexing process should find
it's configuration for the Collection object. This is often a list of
files or directories.\tabularnewline
\bottomrule
\end{longtable}

\begin{longtable}[]{@{}ll@{}}
\toprule
Property & \textbf{ignore.empty.documents}\tabularnewline
Used in &
\href{javadoc/org/terrier/utility/ApplicationSetup.html}{org.terrier.utility.ApplicationSetup},
\href{javadoc/org/terrier/structures/indexing/Indexer.html}{org.terrier.structures.indexing.Indexer}\tabularnewline
Possible values & true, false\tabularnewline
Default value & false\tabularnewline
Configures & IGNORE\_EMPTY\_DOCUMENTS. Whether empty documents have an
entry in the document index.\tabularnewline
\bottomrule
\end{longtable}

\begin{longtable}[]{@{}ll@{}}
\toprule
Property & \textbf{???.process}\tabularnewline
Used in &
\href{javadoc/org/terrier/utility/TagSet.html}{org.terrier.utility.TagSet}\tabularnewline
Possible values & Comma delimited list of tags to process\tabularnewline
Default value & not specified\tabularnewline
Configures & For many of the tokenisers, configures which tags should be
processed. ??? can be TrecDocTags or TrecQueryTags, to configure the
TREC Collection and Query parsers respectively. ??? as FieldTags
specifies the field that should be stored in the index.\tabularnewline
\bottomrule
\end{longtable}

\begin{longtable}[]{@{}ll@{}}
\toprule
Property & \textbf{???.skip}\tabularnewline
Used in &
\href{javadoc/org/terrier/utility/TagSet.html}{org.terrier.utility.TagSet}\tabularnewline
Possible values & Comma delimited list of tags to not
process\tabularnewline
Default value & not specified\tabularnewline
Configures & For many of the tokenisers, configures which tags should be
skipped completely. ??? can be TrecDocTags or TrecQueryTags, to
configure the TREC Collection and Query parsers
respectively.\tabularnewline
\bottomrule
\end{longtable}

\begin{longtable}[]{@{}ll@{}}
\toprule
Property & \textbf{???.doctag}\tabularnewline
Used in &
\href{javadoc/org/terrier/utility/TagSet.html}{org.terrier.utility.TagSet}\tabularnewline
Possible values & Name of tag that marks the start of the document (trec
only)\tabularnewline
Default value & not specified\tabularnewline
Configures & For some of the tokenisers, configures which tag which
contains the opening tag (or query ID). ??? can be TrecDocTags or
TrecQueryTags, to configure the TREC Collection and Query parsers
respectively.\tabularnewline
\bottomrule
\end{longtable}

\begin{longtable}[]{@{}ll@{}}
\toprule
Property & \textbf{???.idtag}\tabularnewline
Used in &
\href{javadoc/org/terrier/utility/TagSet.html}{org.terrier.utility.TagSet}\tabularnewline
Possible values & Name of tag that contains the unique identifier (trec
only)\tabularnewline
Default value & not specified\tabularnewline
Configures & For some of the tokenisers, configures which tag which
contains the document ID (or query ID). ??? can be TrecDocTags or
TrecQueryTags, to configure the TREC Collection and Query parsers
respectively.\tabularnewline
\bottomrule
\end{longtable}

\begin{longtable}[]{@{}ll@{}}
\toprule
Property & \textbf{???.casesensitive}\tabularnewline
Used in &
\href{javadoc/org/terrier/utility/TagSet.html}{org.terrier.utility.TagSet}
\href{javadoc/org/terrier/indexing/TRECCollection.html}{org.terrier.indexing.TRECCollection}\tabularnewline
Possible values & true or false\tabularnewline
Default value & true for TrecDocTags, false otherwise\tabularnewline
Configures & For some of the tokenisers, configures if the tag matching
is case-sensitive or not. The default is true for TRECCollection
(TrecDocTags), and false for FieldTags and TrecQueryTags
(TRECFullTokenizer which is used by the TREC query parser
(TRECQuery)).\tabularnewline
\bottomrule
\end{longtable}

\begin{longtable}[]{@{}ll@{}}
\toprule
Property & \textbf{???.propertytags}\tabularnewline
Used in &
\href{javadoc/org/terrier/utility/TagSet.html}{org.terrier.utility.TagSet}
\href{javadoc/org/terrier/indexing/TRECCollection.html}{org.terrier.indexing.TRECCollection}\tabularnewline
Possible values & Comma delimited list of tags to add as document
properties\tabularnewline
Default value & not specified\tabularnewline
Configures & During indexing this enables document tags to be saved as
document properties instead of being indexed. This is useful to store
document properties in the meta index for use later, e.g. for display by
the Terrier Web-based interface.\tabularnewline
\bottomrule
\end{longtable}

\begin{longtable}[]{@{}ll@{}}
\toprule
Property & \textbf{block.indexing}\tabularnewline
Used in &
\href{javadoc/org/terrier/utility/ApplicationSetup.html}{org.terrier.utility.ApplicationSetup},
\href{javadoc/org/terrier/applications/TRECIndexing.html}{org.terrier.applications.TRECIndexing}\tabularnewline
Possible values & true, false\tabularnewline
Default value & false\tabularnewline
Configures & ApplicationSetup.BLOCK\_INDEXING. Sets whether block
positions should be saved during indexing. This is required to do
phrasal searches. Client code should examine this to determine whether
to use the BasicIndexer or the BlockIndexer.\tabularnewline
\bottomrule
\end{longtable}

\begin{longtable}[]{@{}ll@{}}
\toprule
Property & \textbf{blocks.size}\tabularnewline
Used in &
\href{javadoc/org/terrier/utility/ApplicationSetup.html}{org.terrier.utility.ApplicationSetup},
\href{javadoc/org/terrier/structures/indexing/classical/BasicIndexer.html}{org.terrier.structures.indexing.classical.BasicIndexer}\tabularnewline
Possible values & integer \textgreater{} 0\tabularnewline
Default value & 1\tabularnewline
Configures & ApplicationSetup.BLOCK\_SIZE. The number of terms contained
in the same block\tabularnewline
\bottomrule
\end{longtable}

\begin{longtable}[]{@{}ll@{}}
\toprule
Property & \textbf{blocks.max}\tabularnewline
Used in &
\href{javadoc/org/terrier/utility/ApplicationSetup.html}{org.terrier.utility.ApplicationSetup},
\href{javadoc/org/terrier/structures/indexing/classical/BasicIndexer.html}{org.terrier.structures.indexing.classical.BasicIndexer}\tabularnewline
Possible values & integer \textgreater{}= 0\tabularnewline
Default value & 100000\tabularnewline
Configures & MAX\_BLOCKS. The maximum number of blocks a document may
contain.\tabularnewline
\bottomrule
\end{longtable}

\begin{longtable}[]{@{}ll@{}}
\toprule
Property & \textbf{lowercase}\tabularnewline
Used in &
\href{javadoc/org/terrier/indexing/TaggedDocument.html}{org.terrier.indexing.TaggedDocument},
\href{javadoc/org/terrier/indexing/TRECFullTokenizer.html}{org.terrier.indexing.TRECFullTokenizer}\tabularnewline
Possible values & true, or false\tabularnewline
Default value & true\tabularnewline
Configures & Whether text is converted to lowercase before
parsing\tabularnewline
\bottomrule
\end{longtable}

\begin{longtable}[]{@{}ll@{}}
\toprule
Property & \textbf{tokeniser}\tabularnewline
Used in &
\href{javadoc/org/terrier/indexing/tokenisation/Tokeniser.html}{org.terrier.indexing.tokenisation.Tokeniser}\tabularnewline
Possible values & a classname implementing the Tokeniser
interface\tabularnewline
Default value & EnglishTokeniser\tabularnewline
Configures & The Tokeniser implementation to be used when splitting text
into tokens. This allows for corpora in different languages to be
indexed by setting a Tokeniser implementation appropriate for each
language.\tabularnewline
\bottomrule
\end{longtable}

\begin{longtable}[]{@{}ll@{}}
\toprule
Property & \textbf{indexing.max.tokens}\tabularnewline
Used in &
\href{javadoc/org/terrier/structures/indexing/Indexer.html}{org.terrier.structures.indexing.Indexer}\tabularnewline
Possible values & integer \textgreater{}=0\tabularnewline
Default value & 0\tabularnewline
Configures & Sets a limit to the maximum number of tokens indexed for a
document. The default value 0 means that there is no
limit.\tabularnewline
\bottomrule
\end{longtable}

\begin{longtable}[]{@{}ll@{}}
\toprule
Property & \textbf{indexing.max.docs.per.builder}\tabularnewline
Used in &
\href{javadoc/org/terrier/structures/indexing/Indexer.html}{org.terrier.structures.indexing.Indexer}\tabularnewline
Possible values & integer \textgreater{}=0\tabularnewline
Default value & 18,000,000\tabularnewline
Configures & Sets a limit to the maximum number of documents in one
index during indexing. After this point, a new index will be created,
and at the end, all the indices will be merged. Reasoning: During
classical two-pass indexing, memory is constrained by the TermCodes
table. If too many different unique terms are indexed, then an
OutOfMemoryError will occur. For TREC GOV2 collection, 18 million
documents is a good point to start a new index. The special value 0
means that there is no limit. This property also applies for single-pass
indexing, although it can be safely set higher. It does not apply for
MapReduce indexing.\tabularnewline
\bottomrule
\end{longtable}

\subsubsection{Advanced}\label{advanced}

\begin{longtable}[]{@{}ll@{}}
\toprule
Property & \textbf{termpipelines}\tabularnewline
Used in &
\href{javadoc/org/terrier/querying/Manager.html}{org.terrier.querying.Manager},
\href{javadoc/org/terrier/structures/indexing/Indexer.html}{org.terrier.structures.indexing.Indexer}\tabularnewline
Possible values & Comma delimited list of term pipeline entities to pass
query terms through. Use blank to denote no termpipeline
objects\tabularnewline
Default value & Stopwords,PorterStemmer\tabularnewline
Configures & Defines which term pipeline entities to pass query terms
through.\tabularnewline
\bottomrule
\end{longtable}

\begin{longtable}[]{@{}ll@{}}
\toprule
\begin{minipage}[t]{0.47\columnwidth}\raggedright\strut
Property
\strut\end{minipage} &
\begin{minipage}[t]{0.47\columnwidth}\raggedright\strut
\textbf{invertedfile.processpointers}
\strut\end{minipage}\tabularnewline
\begin{minipage}[t]{0.47\columnwidth}\raggedright\strut
Used in
\strut\end{minipage} &
\begin{minipage}[t]{0.47\columnwidth}\raggedright\strut
\href{javadoc/org/terrier/structures/indexing/classical/InvertedIndexBuilder.html}{org.terrier.structures.indexing.classical.InvertedIndexBuilder},
\href{javadoc/org/terrier/structures/indexing/classical/BlockInvertedIndexBuilder.html}{org.terrier.structures.indexing.classical.BlockInvertedIndexBuilder}
\strut\end{minipage}\tabularnewline
\begin{minipage}[t]{0.47\columnwidth}\raggedright\strut
Possible values
\strut\end{minipage} &
\begin{minipage}[t]{0.47\columnwidth}\raggedright\strut
Integer value \textgreater{} 0
\strut\end{minipage}\tabularnewline
\begin{minipage}[t]{0.47\columnwidth}\raggedright\strut
Default value
\strut\end{minipage} &
\begin{minipage}[t]{0.47\columnwidth}\raggedright\strut
20000000
\strut\end{minipage}\tabularnewline
\begin{minipage}[t]{0.47\columnwidth}\raggedright\strut
Configures
\strut\end{minipage} &
\begin{minipage}[t]{0.47\columnwidth}\raggedright\strut
Defines the number of pointers that should be processed at once when
building the inverted index. The InvertedIndexBuilder first works out
how many terms correspond to that many pointers, then scans the direct
index looking for each of these term, then writes them to inverted
index, then repeats scan for next bunch of terms. Increasing this speeds
up inverted index building for large collections, but uses more memory.
Decrease this if you encounter OutOfMemory errors while building the
inverted index. Note that for block indexing, the default is lower:
2,000,000 pointers.

This option supersedes invertedfile.processterms. For the
invertedfile.processterms strategy to be used, set
invertedfile.processpointers to 0.
\strut\end{minipage}\tabularnewline
\bottomrule
\end{longtable}

\begin{longtable}[]{@{}ll@{}}
\toprule
Property & \textbf{lexicon.builder.merge.lex.max}\tabularnewline
Used in &
\href{javadoc/org/terrier/structures/indexing/LexiconBuilder.html}{org.terrier.structures.indexing.LexiconBuilder},\tabularnewline
Possible values & integer values \textgreater{} 1\tabularnewline
Default value & 16\tabularnewline
Configures & The number of temporary lexicons to merge at once during
indexing. during lexicon building. Bigger is generally faster, but too
many open file-handles causes slowness. 16 is a good trade-off. (See
also the MERGE\_FACTOR in GNU sort source code).\tabularnewline
\bottomrule
\end{longtable}

\begin{longtable}[]{@{}ll@{}}
\toprule
Property & \textbf{indexing.excel.maxfilesize.mb}\tabularnewline
Used in &
\href{javadoc/org/terrier/indexing/MSExcelDocument.html}{org.terrier.indexing.MSExcelDocument}\tabularnewline
Possible values & size of a file in megabytes\tabularnewline
Default value & 0.5\tabularnewline
Configures & The maximum file size of an Excel spreadsheet to be
parsed.\tabularnewline
\bottomrule
\end{longtable}

\begin{longtable}[]{@{}ll@{}}
\toprule
Property &
\textbf{indexing.simplefilecollection.extensionsparsers}\tabularnewline
Used in &
\href{javadoc/org/terrier/indexing/SimpleFileCollection.html}{org.terrier.indexing.SimpleFileCollection}\tabularnewline
Possible values & comma delimited list of file extensions and associated
parsers to use for the corresponding files.\tabularnewline
Default value &
txt:FileDocument,text:FileDocument,tex:FileDocument,bib:FileDocument,
pdf:PDFDocument,html:HTMLDocument,htm:HTMLDocument,xhtml:HTMLDocument,
xml:HTMLDocument,doc:MSWordDocument,ppt:MSPowerpointDocument,xls:MSExcelDocument\tabularnewline
Configures & The parsers to be used for processing files with the
specified extensions.\tabularnewline
\bottomrule
\end{longtable}

\begin{longtable}[]{@{}ll@{}}
\toprule
Property &
\textbf{indexing.simplefilecollection.defaultparser}\tabularnewline
Used in &
\href{javadoc/org/terrier/indexing/SimpleFileCollection.html}{org.terrier.indexing.SimpleFileCollection}\tabularnewline
Possible values & fully qualified class name\tabularnewline
Default value & not specified\tabularnewline
Configures & The parser to use by default for processing files with
unknown extensions\tabularnewline
\bottomrule
\end{longtable}

\begin{longtable}[]{@{}ll@{}}
\toprule
Property & \textbf{trec.blacklist.docids}\tabularnewline
Used in &
\href{javadoc/org/terrier/indexing/TRECCollection.html}{org.terrier.indexing.TRECCollection}\tabularnewline
Possible values & full path to filename\tabularnewline
Default value & not specified\tabularnewline
Configures & The name of a file that contains a black list of document
identifiers to be ignored during indexing\tabularnewline
\bottomrule
\end{longtable}

\begin{longtable}[]{@{}ll@{}}
\toprule
Property & \textbf{trec.collection.class}\tabularnewline
Used in &
\href{javadoc/org/terrier/applications/TRECIndexing.html}{org.terrier.applications.TRECIndexing}\tabularnewline
Possible values & a classname implementing Collection
interface\tabularnewline
Default value & TRECCollection\tabularnewline
Configures & The Collection object to be used to parse the collection.
This allows test collection similar but not identical to TREC to be
parsed using Terrier's TREC tools. New in Terrier 1.1.0 is the ability
to chain Collections. The Collection specified last is the inner-most
one of the chain, the first is the outer-most (i.e. instantiation
right-to-left). the first collection should have a default constructor
(no arguments), while the other collections should take as argument in
their constructor the inner-collection class. E.g.
\texttt{trec.collection.class=RemoveSmallDocsCollection,TRECCollection}.
Instantiation handled by the CollectionFactory class.\tabularnewline
\bottomrule
\end{longtable}

\begin{longtable}[]{@{}ll@{}}
\toprule
Property & \textbf{indexer.meta.forward.keys}\tabularnewline
Used in &
\href{javadoc/org/terrier/structures/indexing/CompressingMetaIndexBuilder.html}{CompressingMetaIndexBuilder}\tabularnewline
Possible values & comma delimited list of properties of a
\href{javadoc/org/terrier/indexing/Document.html}{Document} object that
should be used as metadata.\tabularnewline
Default value & docno\tabularnewline
Configures & The document properties that should be recorded as document
metadata.\tabularnewline
\bottomrule
\end{longtable}

\begin{longtable}[]{@{}ll@{}}
\toprule
Property & \textbf{indexer.meta.forward.keylens}\tabularnewline
Used in &
\href{javadoc/org/terrier/structures/indexing/CompressingMetaIndexBuilder.html}{CompressingMetaIndexBuilder}\tabularnewline
Possible values & comma delimited list of the lengths of the values
corresponding to the keys to be used as document
metadata.\tabularnewline
Default value & 20\tabularnewline
Configures & How long values can be in the MetaIndex.\tabularnewline
\bottomrule
\end{longtable}

\begin{longtable}[]{@{}ll@{}}
\toprule
Property & \textbf{indexer.meta.reverse.keys}\tabularnewline
Used in &
\href{javadoc/org/terrier/structures/indexing/CompressingMetaIndexBuilder.html}{CompressingMetaIndexBuilder}\tabularnewline
Possible values & comma delimited list of the keys that can be used to
uniquely identify documents.\tabularnewline
Default value & 20\tabularnewline
Configures & The MetaIndex keys that can unique identify a document.
E.g. docno,url.\tabularnewline
\bottomrule
\end{longtable}

\begin{longtable}[]{@{}ll@{}}
\toprule
Property & \textbf{max.term.length}\tabularnewline
Used in &
\href{javadoc/org/terrier/utility/ApplicationSetup.html}{org.terrier.utility.ApplicationSetup},
\href{javadoc/org/terrier/indexing/FileDocument.html}{org.terrier.indexing.FileDocument},
\href{javadoc/org/terrier/indexing/TRECFullTokenizer.html}{org.terrier.indexing.TRECFullTokenizer},
\href{javadoc/org/terrier/structures/Lexicon.html}{org.terrier.structures.Lexicon},\tabularnewline
Possible values & Integer value \textgreater{} 0\tabularnewline
Default value & 20\tabularnewline
Configures & MAX\_TERM\_LENGTH. The size in the lexicon reserved for a
string, i.e. the max length of any term in the index.
term.\tabularnewline
\bottomrule
\end{longtable}

\begin{longtable}[]{@{}ll@{}}
\toprule
Property & \textbf{memory.reserved}\tabularnewline
Used in &
\href{javadoc/org/terrier/structures/indexing/singlepass/BasicSinglePassIndexer.html}{org.terrier.structures.indexing.singlepass.BasicSinglePassIndexer}\tabularnewline
Possible values & integer \textgreater{} 0, probably around 50
million\tabularnewline
Default value & 50000000\tabularnewline
Configures & Free memory threshold that forces a run to be committed to
disk in the single-pass indexer. Higher values means less chance of
OutOfMemoryError occurring, but slower indexing speed as more runs will
be generated.\tabularnewline
\bottomrule
\end{longtable}

\begin{longtable}[]{@{}ll@{}}
\toprule
Property & \textbf{memory.heap.usage}\tabularnewline
Used in &
\href{javadoc/org/terrier/structures/indexing/singlepass/BasicSinglePassIndexer.html}{org.terrier.structures.indexing.singlepass.BasicSinglePassIndexer}\tabularnewline
Possible values & positive float, range 0.0f - 1.0f\tabularnewline
Default value & 0.70\tabularnewline
Configures & amount of max heap allocated to JVM before a run is
committed. Smaller values mean more runs and hence slower indexing.
Larger values means more risk of OutOfMemoryError
occurrences.\tabularnewline
\bottomrule
\end{longtable}

\begin{longtable}[]{@{}ll@{}}
\toprule
Property & \textbf{docs.check}\tabularnewline
Used in &
\href{javadoc/org/terrier/structures/indexing/singlepass/BasicSinglePassIndexer.html}{org.terrier.structures.indexing.singlepass.BasicSinglePassIndexer}\tabularnewline
Possible values & positive integer \textgreater{} 0\tabularnewline
Default value & 20\tabularnewline
Configures & how often to check the amount of free memory. Lower values
gives more protection from OutOfMemoryError.\tabularnewline
\bottomrule
\end{longtable}

\begin{longtable}[]{@{}ll@{}}
\toprule
Property & \textbf{inverted2direct.processtokens}\tabularnewline
Used in &
\href{javadoc/org/terrier/structures/indexing/singlepass/Inverted2DirectIndexBuilder.html}{org.terrier.structures.indexing.singlepass.Inverted2DirectIndexBuilder}\tabularnewline
Possible values & positive long \textgreater{} 0\tabularnewline
Default value & 100000000, 10000000 for blocks\tabularnewline
Configures & total number of tokens to attempt each iteration of
building the direct index. Use a lower value if OutOfMemoryError
occurs.\tabularnewline
\bottomrule
\end{longtable}

\begin{longtable}[]{@{}ll@{}}
\toprule
Property &
\textbf{terrier.index.retrievalLoadingProfile.default}\tabularnewline
Used in &
\href{javadoc/org/terrier/structures/Index.html}{org.terrier.structures.Index}\tabularnewline
Possible values & true, false\tabularnewline
Default value & true\tabularnewline
Configures & Index.RETRIEVAL\_LOADING\_PROFILE. Whether index structures
should be preloaded for retrieval.\tabularnewline
\bottomrule
\end{longtable}

\begin{longtable}[]{@{}ll@{}}
\toprule
Property & \textbf{TaggedDocument.abstracts}\tabularnewline
Used in &
\href{javadoc/org/terrier/indexing/TaggedDocument.html}{org.terrier.indexing.TaggedDocument}\tabularnewline
Possible values & Comma delimited list of abstract names to save as
document properties\tabularnewline
Default value & not specified\tabularnewline
Configures & The list of abstract names to save as document properties
when indexing a TaggedDocument or one of its subclasses.\tabularnewline
\bottomrule
\end{longtable}

\begin{longtable}[]{@{}ll@{}}
\toprule
Property & \textbf{TaggedDocument.abstracts.tags}\tabularnewline
Used in &
\href{javadoc/org/terrier/indexing/TaggedDocument.html}{org.terrier.indexing.TaggedDocument}\tabularnewline
Possible values & Comma delimited list of tags from which to save
abstracts\tabularnewline
Default value & not specified\tabularnewline
Configures & The names of tags to save text from. ELSE is special tag
name, which means anything not consumed by other tags.\tabularnewline
\bottomrule
\end{longtable}

\begin{longtable}[]{@{}ll@{}}
\toprule
Property &
\textbf{TaggedDocument.abstracts.tags.casesensitive}\tabularnewline
Used in &
\href{javadoc/org/terrier/indexing/TaggedDocument.html}{org.terrier.indexing.TaggedDocument}\tabularnewline
Possible values & true or false\tabularnewline
Default value & false\tabularnewline
Configures & Configures if the tag matching is case-sensitive or
not.\tabularnewline
\bottomrule
\end{longtable}

\begin{longtable}[]{@{}ll@{}}
\toprule
Property & \textbf{TaggedDocument.abstracts.lengths}\tabularnewline
Used in &
\href{javadoc/org/terrier/indexing/TaggedDocument.html}{org.terrier.indexing.TaggedDocument}\tabularnewline
Possible values & Comma delimited list of maximum lengths for each
abstract\tabularnewline
Default value & Length 0\tabularnewline
Configures & The max lengths of the abstracts. Defaults to
empty.\tabularnewline
\bottomrule
\end{longtable}

\begin{longtable}[]{@{}ll@{}}
\toprule
Property & \textbf{FileDocument.abstract}\tabularnewline
Used in &
\href{javadoc/org/terrier/indexing/FileDocument.html}{org.terrier.indexing.FileDocument}\tabularnewline
Possible values & Name to call the abstract\tabularnewline
Default value & not specified\tabularnewline
Configures & The name of the abstract to save from the document. Note
that only if this is set will an abstract be generated. Only a single
abstract can be generated from a FileDocument.\tabularnewline
\bottomrule
\end{longtable}

\begin{longtable}[]{@{}ll@{}}
\toprule
Property & \textbf{FileDocument.abstract.length}\tabularnewline
Used in &
\href{javadoc/org/terrier/indexing/FileDocument.html}{org.terrier.indexing.FileDocument}\tabularnewline
Possible values & The maximum length for the abstract\tabularnewline
Default value & 0\tabularnewline
Configures & The maximum length for the abstract.\tabularnewline
\bottomrule
\end{longtable}

\hypertarget{retrieval}{\subsection{\texorpdfstring{\href{}{Retrieval}}{Retrieval}}\label{retrieval}}

\subsubsection{Model}\label{model}

\begin{longtable}[]{@{}ll@{}}
\toprule
Property & \textbf{ignore.low.idf.terms}\tabularnewline
Used in &
\href{javadoc/org/terrier/matching/Matching.html}{org.terrier.matching.Matching}\tabularnewline
Possible values & true, false\tabularnewline
Default value & true\tabularnewline
Configures & Ignores a term that has a low IDF, ie appears in many
documents. You may wish to turn this off for small or focused
collections.\tabularnewline
\bottomrule
\end{longtable}

\subsubsection{Interactive Retrieval}\label{interactive-retrieval}

\begin{longtable}[]{@{}ll@{}}
\toprule
Property & \textbf{interactive.output.format.length}\tabularnewline
Used in &
\href{javadoc/org/terrier/applications/InteractiveQuerying.html}{org.terrier.applications.InteractiveQuerying}\tabularnewline
Possible values & integer number \textgreater{} 0\tabularnewline
Default value & 1000\tabularnewline
Configures & the maximum number of results to be displayed for
Interactive querying\tabularnewline
\bottomrule
\end{longtable}

\subsubsection{TREC-style Batch
Retrieval}\label{trec-style-batch-retrieval}

\begin{longtable}[]{@{}ll@{}}
\toprule
Property & \textbf{trec.model}\tabularnewline
Used in &
\href{javadoc/org/terrier/applications/batchquerying/TRECQuerying.html}{org.terrier.applications.batchquerying.TRECQuerying}\tabularnewline
Possible values & Name of weighting models\tabularnewline
Default value & InL2\tabularnewline
Configures & The weighting model to use during retrieval.\tabularnewline
\bottomrule
\end{longtable}

\begin{longtable}[]{@{}ll@{}}
\toprule
Property & \textbf{trec.results}\tabularnewline
Used in &
\href{javadoc/org/terrier/utility/ApplicationSetup.html}{org.terrier.utility.ApplicationSetup}\tabularnewline
Possible values & Absolute directory path\tabularnewline
Default value & TERRIER\_VAR + value of ``trec.results''"\tabularnewline
Configures & TREC\_RESULTS. Where TREC*Querying applications should
store their results files and where evaluation files should be
placed.\tabularnewline
\bottomrule
\end{longtable}

\begin{longtable}[]{@{}ll@{}}
\toprule
Property & \textbf{trec.results.file}\tabularnewline
Used in &
\href{javadoc/org/terrier/applications/batchquerying/TRECQuerying.html}{org.terrier.applications.batchquerying.TRECQuerying}\tabularnewline
Possible values & A valid file name.\tabularnewline
Default value & not specified\tabularnewline
Configures & An arbitrary name for a TREC results file.\tabularnewline
\bottomrule
\end{longtable}

\begin{longtable}[]{@{}ll@{}}
\toprule
Property & \textbf{trec.querycounter.type}\tabularnewline
Used in &
\href{javadoc/org/terrier/applications/batchquerying/TRECQuerying.html}{org.terrier.applications.batchquerying.TRECQuerying}\tabularnewline
Possible values & sequential, random\tabularnewline
Default value & sequential\tabularnewline
Configures & Whether to use sequential (auto-incremented) or randomly
generated suffixes for run names.\tabularnewline
\bottomrule
\end{longtable}

\begin{longtable}[]{@{}ll@{}}
\toprule
Property & \textbf{trec.results.suffix}\tabularnewline
Used in &
\href{javadoc/org/terrier/utility/ApplicationSetup.html}{org.terrier.utility.ApplicationSetup}\tabularnewline
Possible values & string\tabularnewline
Default value & .res\tabularnewline
Configures & ApplicationSetup.TREC\_RESULTS\_SUFFIX. The suffix to be
used for result files.\tabularnewline
\bottomrule
\end{longtable}

\begin{longtable}[]{@{}ll@{}}
\toprule
Property & \textbf{trec.runtag}\tabularnewline
Used in &
\href{javadoc/org/terrier/applications/batchquerying/TRECQuerying.html}{org.terrier.applications.batchquerying.TRECQuerying}\tabularnewline
Possible values & string\tabularnewline
Default value & not specified\tabularnewline
Configures & An arbitrary runtag (6th field) for a TREC results
file.\tabularnewline
\bottomrule
\end{longtable}

\begin{longtable}[]{@{}ll@{}}
\toprule
Property & \textbf{trec.topics}\tabularnewline
Used in &
\href{javadoc/org/terrier/applications/batchquerying/TRECQuerying.html}{org.terrier.applications.batchquerying.TRECQuerying}\tabularnewline
Possible values & A valid topics file name\tabularnewline
Default value & not specified\tabularnewline
Configures & A single file containing the topics to be
processed.\tabularnewline
\bottomrule
\end{longtable}

\begin{longtable}[]{@{}ll@{}}
\toprule
Property & \textbf{trec.topics.parser}\tabularnewline
Used in &
\href{javadoc/org/terrier/applications/batchquerying/TRECQuerying.html}{org.terrier.applications.batchquerying.TRECQuerying}\tabularnewline
Possible values & A sub-class of
org.terrier.structures.QuerySource\tabularnewline
Default value & TRECQuery\tabularnewline
Configures & The class to be used when parsing a topics
file.\tabularnewline
\bottomrule
\end{longtable}

\begin{longtable}[]{@{}ll@{}}
\toprule
Property & \textbf{trec.encoding}\tabularnewline
Used in &
\href{javadoc/org/terrier/structures/TRECQuery.html}{org.terrier.structures.TRECQuery},
\href{javadoc/org/terrier/indexing/TRECCollection.html}{org.terrier.indexing.TRECCollection},
\href{javadoc/org/terrier/indexing/TRECUTFCollection.html}{org.terrier.indexing.TRECUTFCollection},
\href{javadoc/org/terrier/terms/Stopwords.html}{org.terrier.terms.Stopwords}\tabularnewline
Possible values & A valid encoding scheme.\tabularnewline
Default value & The system's default charset.\tabularnewline
Configures & The encoding to use for topics, documents, and stopwords
files.\tabularnewline
\bottomrule
\end{longtable}

\begin{longtable}[]{@{}ll@{}}
\toprule
Property & \textbf{trec.qrels}\tabularnewline
Used in &
\href{javadoc/org/terrier/utility/ApplicationSetup.html}{org.terrier.utility.ApplicationSetup}\tabularnewline
Possible values & Absolute filename\tabularnewline
Default value & not specified\tabularnewline
Configures & A single file containing the qrels to evaluate
with.\tabularnewline
\bottomrule
\end{longtable}

\begin{longtable}[]{@{}ll@{}}
\toprule
Property & \textbf{trec.output.format.length}\tabularnewline
Used in &
\href{javadoc/org/terrier/applications/batchquerying/TRECQuerying.html}{org.terrier.applications.batchquerying.TRECQuerying},\tabularnewline
Possible values & integer number \textgreater{} 0\tabularnewline
Default value & 1000\tabularnewline
Configures & the maximum number of results to be displayed for TREC
querying\tabularnewline
\bottomrule
\end{longtable}

\begin{longtable}[]{@{}ll@{}}
\toprule
Property & \textbf{trec.querying.outputformat}\tabularnewline
Used in &
\href{javadoc/org/terrier/applications/batchquerying/TRECQuerying.html}{org.terrier.applications.batchquerying.TRECQuerying}\tabularnewline
Possible values & A sub-class of
TRECQuerying\$OutputFormat\tabularnewline
Default value & TRECQuerying\$TRECDocnoOutputFormat\tabularnewline
Configures & The class used to write the results file.\tabularnewline
\bottomrule
\end{longtable}

\begin{longtable}[]{@{}ll@{}}
\toprule
Property & \textbf{trec.querying.resultscache}\tabularnewline
Used in &
\href{javadoc/org/terrier/applications/batchquerying/TRECQuerying.html}{org.terrier.applications.batchquerying.TRECQuerying}\tabularnewline
Possible values & A sub-class of
TRECQuerying\$QueryResultCache\tabularnewline
Default value & TRECQuerying\$NullQueryResultCache\tabularnewline
Configures & The class used to cache the results.\tabularnewline
\bottomrule
\end{longtable}

\begin{longtable}[]{@{}ll@{}}
\toprule
Property & \textbf{trec.querying.dump.settings}\tabularnewline
Used in &
\href{javadoc/org/terrier/applications/batchquerying/TRECQuerying.html}{org.terrier.applications.batchquerying.TRECQuerying}\tabularnewline
Possible values & true, false\tabularnewline
Default value & true\tabularnewline
Configures & Whether the settings used to generate a results file should
be dumped to a .settings file in conjunction with the .res
file.\tabularnewline
\bottomrule
\end{longtable}

\begin{longtable}[]{@{}ll@{}}
\toprule
Property & \textbf{trec.iteration}\tabularnewline
Used in &
\href{javadoc/org/terrier/applications/batchquerying/TRECQuerying.html}{org.terrier.applications.batchquerying.TRECQuerying}\tabularnewline
Possible values & String\tabularnewline
Default value & Q\tabularnewline
Configures & Related to standard format of TREC results\tabularnewline
\bottomrule
\end{longtable}

\begin{longtable}[]{@{}ll@{}}
\toprule
Property & \textbf{trec.manager}\tabularnewline
Used in &
\href{javadoc/org/terrier/applications/batchquerying/TRECQuerying.html}{org.terrier.applications.batchquerying.TRECQuerying},\tabularnewline
Possible values & String, Class name in
org.terrier.querying\tabularnewline
Default value & Manager\tabularnewline
Configures & The Manager class to use during querying\tabularnewline
\bottomrule
\end{longtable}

\begin{longtable}[]{@{}ll@{}}
\toprule
Property & \textbf{trec.matching}\tabularnewline
Used in &
\href{javadoc/org/terrier/applications/batchquerying/TRECQuerying.html}{org.terrier.applications.batchquerying.TRECQuerying},\tabularnewline
Possible values & String, Class name in
org.terrier.matching\tabularnewline
Default value & org.terrier.matching.taat.Full\tabularnewline
Configures & The Matching class to use during querying\tabularnewline
\bottomrule
\end{longtable}

\begin{longtable}[]{@{}ll@{}}
\toprule
Property & \textbf{matching.trecresults.file}\tabularnewline
Used in &
\href{javadoc/org/terrier/matching/TRECResultsMatching.html}{org.terrier.matching.TRECResultsMatching}\tabularnewline
Possible values & A valid TREC results file\tabularnewline
Default value & not specified\tabularnewline
Configures & The TREC-formatted results file containing search results
for each of the topics specified in the \texttt{trec.topics}
property\tabularnewline
\bottomrule
\end{longtable}

\begin{longtable}[]{@{}ll@{}}
\toprule
Property & \textbf{matching.trecresults.format}\tabularnewline
Used in &
\href{javadoc/org/terrier/matching/TRECResultsMatching.html}{org.terrier.matching.TRECResultsMatching}\tabularnewline
Possible values & \texttt{DOCNO}, \texttt{DOCID}\tabularnewline
Default value & \texttt{DOCNO}\tabularnewline
Configures & Whether the TREC-formatted results file contains DOCNOs or
Terrier's internal (integer) docids\tabularnewline
\bottomrule
\end{longtable}

\begin{longtable}[]{@{}ll@{}}
\toprule
Property & \textbf{matching.trecresults.scores}\tabularnewline
Used in &
\href{javadoc/org/terrier/matching/TRECResultsMatching.html}{org.terrier.matching.TRECResultsMatching}\tabularnewline
Possible values & true, false\tabularnewline
Default value & true\tabularnewline
Configures & Whether Terrier should use the relevance scores from the
TREC-formatted results file\tabularnewline
\bottomrule
\end{longtable}

\begin{longtable}[]{@{}ll@{}}
\toprule
Property & \textbf{matching.trecresults.length}\tabularnewline
Used in &
\href{javadoc/org/terrier/matching/TRECResultsMatching.html}{org.terrier.matching.TRECResultsMatching}\tabularnewline
Possible values & a non-negative integer\tabularnewline
Default value & 1000\tabularnewline
Configures & The maximum number of results to be retrieved from a TREC
results file for each query. If set to 0, all available results are
retrieved (note that setting this property to 0 may slow down the
retrieval process for large collections, as a result set of the size of
the collection will be allocated in memory)\tabularnewline
\bottomrule
\end{longtable}

\subsubsection{Query Expansion}\label{query-expansion}

\begin{longtable}[]{@{}ll@{}}
\toprule
Property & \textbf{parameter.free.expansion}\tabularnewline
Used in &
\href{javadoc/org/terrier/matching/models/queryexpansion/QueryExpansionModel.html}{org.terrier.matching.models.queryexpansion.QueryExpansionModel}\tabularnewline
Possible values & true or false\tabularnewline
Default value & true\tabularnewline
Configures & Whether we apply parameter-free query expansion or
not.\tabularnewline
\bottomrule
\end{longtable}

\begin{longtable}[]{@{}ll@{}}
\toprule
Property & \textbf{rocchio.beta}\tabularnewline
Used in &
\href{javadoc/org/terrier/matching/models/queryexpansion/QueryExpansionModel.html}{org.terrier.matching.models.queryexpansion.QueryExpansionModel}\tabularnewline
Possible values & float\tabularnewline
Default value & 0.4\tabularnewline
Configures & The parameter of Rocchio's automatic query
expansion\tabularnewline
\bottomrule
\end{longtable}

\begin{longtable}[]{@{}ll@{}}
\toprule
Property & \textbf{trec.qe.model}\tabularnewline
Used in &
\href{javadoc/org/terrier/applications/batchquerying/TRECQuerying.html}{org.terrier.applications.batchquerying.TRECQuerying}\tabularnewline
Possible values & Query expansion models\tabularnewline
Default value & Bo1\tabularnewline
Configures & A name of a query expansion model\tabularnewline
\bottomrule
\end{longtable}

\begin{longtable}[]{@{}ll@{}}
\toprule
Property & \textbf{expansion.documents}\tabularnewline
Used in &
\href{javadoc/org/terrier/matching/models/queryexpansion/QueryExpansionModel.html}{org.terrier.matching.models.queryexpansion.QueryExpansionModel}\tabularnewline
Possible values & integer\tabularnewline
Default value & 3\tabularnewline
Configures & The number of top-ranked documents to be considered in the
pseudo relevance set\tabularnewline
\bottomrule
\end{longtable}

\begin{longtable}[]{@{}ll@{}}
\toprule
Property & \textbf{expansion.terms}\tabularnewline
Used in &
\href{javadoc/org/terrier/matching/models/queryexpansion/QueryExpansionModel.html}{org.terrier.matching.models.queryexpansion.QueryExpansionModel},\tabularnewline
Possible values & integer\tabularnewline
Default value & 10\tabularnewline
Configures & The number of the highest weighted terms from the pseudo
relevance set to be added to the original query. There can be overlap
between the original query terms and the added terms from the pseudo
relevance set\tabularnewline
\bottomrule
\end{longtable}

\begin{longtable}[]{@{}ll@{}}
\toprule
Property & \textbf{expansion.mindocuments}\tabularnewline
Used in &
\href{javadoc/org/terrier/querying/ExpansionTerms.html}{org.terrier.querying.ExpansionTerms}\tabularnewline
Possible values & integer\tabularnewline
Default value & 2\tabularnewline
Configures & The minimum number of documents a term must exist in before
it can be considered to be informative. Defaults to 2. For more
information, see Giambattista Amati: Information Theoretic Approach to
Information Extraction. FQAS 2006: 519-529
\href{http://dx.doi.org/10.1007/11766254_44}{DOI
10.1007/11766254\_44}\tabularnewline
\bottomrule
\end{longtable}

\begin{longtable}[]{@{}ll@{}}
\toprule
Property & \textbf{qe.feedback.selector}\tabularnewline
Used in &
\href{javadoc/org/terrier/querying/QueryExpansion.html}{org.terrier.querying.QueryExpansion}\tabularnewline
Possible values & classname, or comma-delimited class
names\tabularnewline
Default value & PseudoRelevanceFeedbackSelector\tabularnewline
Configures & Class(es) that select feedback documents for query
expansion. All classes must implement
\href{javadoc/org/terrier/querying/FeedbackSelector.html}{FeedbackSelector}.
If more than one is specified, then a chain is assumed, with last being
innermost in the chain.\tabularnewline
\bottomrule
\end{longtable}

\begin{longtable}[]{@{}ll@{}}
\toprule
Property & \textbf{qe.expansion.terms.class}\tabularnewline
Used in &
\href{javadoc/org/terrier/querying/QueryExpansion.html}{org.terrier.querying.QueryExpansion}\tabularnewline
Possible values & classname, or comma-delimited class
names\tabularnewline
Default value & DFRBagExpansionTerms\tabularnewline
Configures & Class(es) that select terms during query expansion. All
classes must extend
\href{javadoc/org/terrier/querying/ExpansionTerms.html}{ExpansionTerms}.
If more than one is specified, then a chain is assumed, with last being
innermost in the chain.\tabularnewline
\bottomrule
\end{longtable}

\subsubsection{Querying}\label{querying}

\begin{longtable}[]{@{}ll@{}}
\toprule
Property & \textbf{match.empty.query}\tabularnewline
Used in &
\href{javadoc/org/terrier/matching/Matching.html}{org.terrier.matching.Matching}\tabularnewline
Possible values & true, false\tabularnewline
Default value & true\tabularnewline
Configures & If true, return all documents for an empty query. Use this
if you have post filter/processes to filter out the documents. E.g.
link: site: etc\tabularnewline
\bottomrule
\end{longtable}

\begin{longtable}[]{@{}ll@{}}
\toprule
Property & \textbf{querying.allowed.controls}\tabularnewline
Used in &
\href{javadoc/org/terrier/querying/Manager.html}{org.terrier.querying.Manager}\tabularnewline
Possible values & Comma delimited list of which controls are allowed to
be specified on the query. For use in interactive
querying.\tabularnewline
Default value & c, range\tabularnewline
Configures & Comma delimited list of which controls are allowed to be
specified on the query. For use in interactive querying.
``String:String'' in the query are assumed to be fields unless the first
string is an allowed control. An example value would be: c, range, link,
site.\tabularnewline
\bottomrule
\end{longtable}

\begin{longtable}[]{@{}ll@{}}
\toprule
Property & \textbf{querying.default.controls}\tabularnewline
Used in &
\href{javadoc/org/terrier/querying/Manager.html}{org.terrier.querying.Manager}\tabularnewline
Possible values & Comma delimited list of control names and values.
Names and values are separated by colon.\tabularnewline
Default value & not specified\tabularnewline
Configures & Sets the defaults control values for the querying process.
Controls are used to control the querying process, and may be used to
set matching models, post filters post processes etc. An example value
would be: c:10,site:gla.ac.uk\tabularnewline
\bottomrule
\end{longtable}

\begin{longtable}[]{@{}ll@{}}
\toprule
Property & \textbf{querying.postprocesses.order}\tabularnewline
Used in &
\href{javadoc/org/terrier/querying/Manager.html}{org.terrier.querying.Manager}\tabularnewline
Possible values & Comma delimited list of all allowed post
processes.\tabularnewline
Default value & not specified\tabularnewline
Configures & Specifies the order in which post processes may be be
called, and those that may be called. This is because post processes
often have inter-dependencies. An example value would be:
QueryExpansion,Scope,Site\tabularnewline
\bottomrule
\end{longtable}

\begin{longtable}[]{@{}ll@{}}
\toprule
Property & \textbf{querying.postprocesses.controls}\tabularnewline
Used in &
\href{javadoc/org/terrier/querying/Manager.html}{org.terrier.querying.Manager}\tabularnewline
Possible values & Comma and colon delimited list of control names and
post process names.\tabularnewline
Default value & not specified\tabularnewline
Configures & Specifies which controls enable which post processes. An
example value would be:
site:Site,qe:QueryExpansion,scope:Scope\tabularnewline
\bottomrule
\end{longtable}

\begin{longtable}[]{@{}ll@{}}
\toprule
Property & \textbf{querying.postfilters.order}\tabularnewline
Used in &
\href{javadoc/org/terrier/querying/Manager.html}{org.terrier.querying.Manager}\tabularnewline
Possible values & Comma delimited list of all allowed post
filters.\tabularnewline
Default value & not specified\tabularnewline
Configures & Specifies the order in which post filters may be be called,
and those that may be called. This is because post filters often have
inter-dependencies. An example value would be: LinkFilter\tabularnewline
\bottomrule
\end{longtable}

\begin{longtable}[]{@{}ll@{}}
\toprule
Property & \textbf{querying.postfilters.controls}\tabularnewline
Used in &
\href{javadoc/org/terrier/querying/Manager.html}{org.terrier.querying.Manager}\tabularnewline
Possible values & Comma and colon delimited list of control names and
post filter names.\tabularnewline
Default value & not specified\tabularnewline
Configures & Specifies which controls enable which post filters. An
example value would be: link:LinkFilter\tabularnewline
\bottomrule
\end{longtable}

\subsubsection{Advanced}\label{advanced-1}

\begin{longtable}[]{@{}ll@{}}
\toprule
Property & \textbf{matching.dsms}\tabularnewline
Used in &
\href{javadoc/org/terrier/matching/Matching.html}{org.terrier.matching.Matching}\tabularnewline
Possible values & Comma delimited names of classes in
uk/ac/gla/terrier/matching/dsms, or other fully qualified
models\tabularnewline
Default value & not specified\tabularnewline
Configures & Specifies the static
\href{javadoc/org/terrier/matching/dsms/DocumentScoreModifier.html}{org.terrier.matching.dsms.DocumentScoreModifier}s
that should be applied to all terms of all queries.\tabularnewline
\bottomrule
\end{longtable}

\begin{longtable}[]{@{}ll@{}}
\toprule
Property & \textbf{matching.retrieved\_set\_size}\tabularnewline
Used in &
\href{javadoc/org/terrier/matching/Matching.html}{org.terrier.matching.Matching}\tabularnewline
Possible values & integer values \textgreater{} 0\tabularnewline
Default value & 1000\tabularnewline
Configures & Maximum size of the result set.\tabularnewline
\bottomrule
\end{longtable}

\subsection{\texorpdfstring{\href{}{Desktop
Terrier}}{Desktop Terrier}}\label{desktop-terrier}

\begin{longtable}[]{@{}ll@{}}
\toprule
Property & \textbf{desktop.file.associations}\tabularnewline
Used in &
\href{javadoc/org/terrier/applications/desktop/filehandling/AssociationFileOpener.html}{org.terrier.applications.desktop.filehandling.AssociationFileOpener}\tabularnewline
Possible values & absolute path to filename\tabularnewline
Default value & TERRIER\_VAR/desktop.fileassoc\tabularnewline
Configures & the name of the file in which we save the file type
associations with applications. If no absolute path is specified it will
be presumed by TERRIER\_HOME/var\tabularnewline
\bottomrule
\end{longtable}

\begin{longtable}[]{@{}ll@{}}
\toprule
Property & \textbf{desktop.indexing.singlepass}\tabularnewline
Used in &
\href{javadoc/org/terrier/applications/desktop/DesktopTerrier.html}{org.terrier.applications.desktop.DesktopTerrier}\tabularnewline
Possible values & true, false\tabularnewline
Default value & false\tabularnewline
Configures & Whether single-pass indexing is used by in the Desktop
Terrier.\tabularnewline
\bottomrule
\end{longtable}

Property

\textbf{desktop.directories.spec}

Used in

\href{javadoc/org/terrier/applications/desktop/DesktopTerrier.html}{org.terrier.applications.desktop.DesktopTerrier}

Possible values

absolute path to filename

Default value

TERRIER\_VAR/desktop.spec

Configures

the name of the file that holds a list of directories that are to be
indexed by the Desktop Terrier application

\begin{longtable}[]{@{}ll@{}}
\toprule
Property & \textbf{desktop.directories.filelist}\tabularnewline
Used in &
\href{javadoc/org/terrier/applications/desktop/DesktopTerrier.html}{org.terrier.applications.desktop.DesktopTerrier}\tabularnewline
Possible values & absolute path to filename\tabularnewline
Default value &
TERRIER\_VAR\textbackslash{}index\textbackslash{}data.filelist\tabularnewline
Configures & the name of the file in which we list all files that have
been indexed\tabularnewline
\bottomrule
\end{longtable}

\subsection{\texorpdfstring{\href{}{Miscellaneous}}{Miscellaneous}}\label{miscellaneous}

\begin{longtable}[]{@{}ll@{}}
\toprule
Property & \textbf{stopwords.intern.terms}\tabularnewline
Used in &
\href{javadoc/org/terrier/terms/Stopwords.html}{org.terrier.terms.Stopwords}\tabularnewline
Possible values & true, false\tabularnewline
Default value & false\tabularnewline
Configures & Whether stopwords should be
\href{http://java.sun.com/j2se/1.5.0/docs/api/java/lang/String.html\#intern()}{interned}
during indexing.\tabularnewline
\bottomrule
\end{longtable}

{[}\href{hadoop_indexing.html}{Previous: Hadoop MapReduce Indexing with
Terrier}{]} {[}\href{index.html}{Contents}{]}
{[}\href{terrier_develop.html}{Next: Developing with Terrier}{]}

\begin{center}\rule{0.5\linewidth}{\linethickness}\end{center}

Webpage: \url{http://terrier.org}\\
Contact:
\href{mailto:terrier@dcs.gla.ac.uk}{\nolinkurl{terrier@dcs.gla.ac.uk}}\\
\href{http://www.dcs.gla.ac.uk/}{School of Computing Science}\\
Copyright (C) 2004-2015 \href{http://www.gla.ac.uk/}{University of
Glasgow}
