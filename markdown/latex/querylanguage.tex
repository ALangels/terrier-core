{[}\href{learning.html}{Previous: Learning to Rank with Terrier}{]}
{[}\href{index.html}{Contents}{]} {[}\href{evaluation.html}{Next:
Evaluation of Experiments}{]}\\

\section{Query Language}\label{query-language}

Terrier offers a flexible and powerful query language for searching with
phrases, fields, or specifying that terms are required to appear in the
retrieved documents. Some examples of queries are the following:

\begin{longtable}[]{@{}ll@{}}
\toprule
term1 term2 & retrieves documents that contain term1 or term2 (they do
not need to contain both of them).\tabularnewline
\{term1 term2\} & retrieves documents that contain term1 or term2, where
they are treated as synonyms of each other (they do not need to contain
both of them).\tabularnewline
term1\^{}2.3 & the weight of term1 is multiplied by 2.3.\tabularnewline
+term1 +term2 & retrieves documents that contain both term1 and
term2.\tabularnewline
+term1 -term2 & retrieves documents that contain term1 and do not
contain term2.\tabularnewline
title:term1 & retrieves documents that contain term1 in the title field
(\href{configure_indexing.html\#fields}{Field indexing} must be
configured to record the title field).\tabularnewline
title:(term1 term2) & retrieves documents that contain term1 or term2 in
the title field (\href{configure_indexing.html\#fields}{Field indexing}
must be configured to record the title field).\tabularnewline
term1 -title:term2 & retrieves documents that contain term1, but must
not contain term2 in the title field.\tabularnewline
``term1 term2'' & retrieves documents where the terms term1 and term2
appear in a phrase.\tabularnewline
``term1 term2''\textasciitilde{}n & retrieves documents where the terms
term1 and term2 appear within a distance of n blocks. The order of the
terms is not considered.\tabularnewline
\bottomrule
\end{longtable}

Combinations of the different constructs are possible as well. For
example, the query \texttt{term1\ term2\ -"term1\ term2"} would retrieve
all the documents that contain at least one of the terms term1 and
term2, but not the documents where the phrase ``term1 term2'' appears.

Note that in some configurations, the Terrier query language may not be
available by default. In particular, if batch processing queries from a
file using a class that extends
\href{javadoc/org/terrier/applications/batchquerying/TRECQuery.html}{TRECQuery},
then the queries are pre-processed by a tokeniser that may remove the
query language characters (e.g. brackets and colons). To use the Terrier
query language in this case, you should use
\href{javadoc/org/terrier/applications/batchquerying/SingleLineTRECQuery.html}{SingleLineTRECQuery}
and set \texttt{SingleLineTRECQuery.tokenise} to false in the
\texttt{terrier.properties} file.

{[}\href{learning.html}{Previous: Learning to Rank with Terrier}{]}
{[}\href{index.html}{Contents}{]} {[}\href{evaluation.html}{Next:
Evaluation of Experiments}{]}

\begin{center}\rule{0.5\linewidth}{\linethickness}\end{center}

Webpage: \url{http://terrier.org}\\
Contact:
\href{mailto:terrier@dcs.gla.ac.uk}{\nolinkurl{terrier@dcs.gla.ac.uk}}\\
\href{http://www.dcs.gla.ac.uk/}{School of Computing Science}\\
Copyright (C) 2004-2015 \href{http://www.gla.ac.uk/}{University of
Glasgow}. All Rights Reserved.
