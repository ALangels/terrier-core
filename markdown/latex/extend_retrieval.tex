{[}\href{extend_indexing.html}{Previous: Extending Indexing}{]}
{[}\href{index.html}{Contents}{]} {[}\href{compression.html}{Next:
Pluggable Compression}{]}\\

\section{Extending Retrieval in
Terrier}\label{extending-retrieval-in-terrier}

\subsection{Altering the retrieval
process}\label{altering-the-retrieval-process}

It is very easy to alter the retrieval process in Terrier, as there are
many ``hooks'' at which external classes can be involved. Firstly, you
are free when writing your own application to render the results from
Terrier in your own way. Results in Terrier come in the form of a
\href{javadoc/org/terrier/matching/ResultSet.html}{ResultSet}.

An application's interface with Terrier is through the
\href{javadoc/org/terrier/querying/Manager.html}{Manager} class. The
manager firstly pre-processes the query, by applying it to the
configured
\href{javadoc/org/terrier/terms/TermPipeline.html}{TermPipeline}. Then
it calls the \href{}{Matching} class, which is responsible for matching
documents to the query, and scoring the documents using a
\href{javadoc/org/terrier/matching/models/WeightingModel.html}{WeightingModel}.
Internally, Matching implementations use the
\href{javadoc/org/terrier/matching/PostingListManager.html}{PostingListManager}
to open an
\href{javadoc/org/terrier/structures/postings/IterablePosting.html}{IterablePosting}
for each query term. The overall score of a document to the entire query
can be modified by using a
\href{javadoc/org/terrier/matching/dsms/DocumentScoreModifier.html}{DocumentScoreModifier},
which can be set by the \texttt{matching.dsms} property.

Once the \href{javadoc/org/terrier/matching/ResultSet.html}{ResultSet}
has been returned to the
\href{javadoc/org/terrier/querying/Manager.html}{Manager}, there are two
further phases, namely
\href{javadoc/org/terrier/querying/PostProcess.html}{PostProcessing} and
\href{javadoc/org/terrier/querying/PostFilter.html}{PostFiltering}. In
PostProcessing, the ResultSet can be altered in any way - for example,
\href{javadoc/org/terrier/querying/QueryExpansion.html}{QueryExpansion}
expands the query, and then calls Matching again to generate an improved
ranking of documents. PostFiltering is simpler, allowing documents to be
either included or excluded - this is ideal for interactive applications
where users want to restrict the domain of the documents being
retrieved.

\subsection{Changing Batch Retrieval}\label{changing-batch-retrieval}

\href{javadoc/org/terrier/applications/batchquerying/TRECQuerying.html}{TRECQuerying}
is the main way in which retrieval is deployed for batch retrieval
experiments. It has a multitude of ways in which it can be extended:

\begin{itemize}
\tightlist
\item
  \textbf{Format of input topics}: Terrier supports topics in two
  formats (\href{javadoc/org/terrier/structures/TRECQuery.html}{TREC
  tagged}, or
  \href{javadoc/org/terrier/structures/SingleLineTRECQuery.html}{one
  query per line}). If neither of these is suitable, then you can
  implement another
  \href{javadoc/org/terrier/applications/batchquerying/QuerySource.html}{QuerySource}
  that knows how parse your topics files. Use the
  \texttt{trec.topics.parser} property to configure Terrier to use your
  new QuerySource. E.g.
  \texttt{trec.topics.parser=my.package.DBTopicsSource}.
\item
  \textbf{Format of output results}: You can implement another
  \href{javadoc/org/terrier/structures/outputformat/OutputFormat.html}{OutputFormat}
  to change the format of the results in the .res files. Use the
  \texttt{trec.querying.outputformat} property to configure Terrier to
  use your new OutputFormat. E.g.
  \texttt{trec.querying.outputformat=my.package.MyTRECResultsFormat}.
\end{itemize}

\subsection{Altering query expansion}\label{altering-query-expansion}

\href{javadoc/org/terrier/querying/QueryExpansion.html}{QueryExpansion}
has various ways in which it can be extended:

\begin{itemize}
\tightlist
\item
  To change the exact formula used to score occurrences, implement
  \href{javadoc/org/terrier/matching/models/queryexpansion/QueryExpansionModel.html}{QueryExpansionModel}.
\item
  Currently, terms are weighted from the entire feedback set as one
  large ``bag of words''. To change this, extend
  \href{javadoc/org/terrier/querying/ExpansionTerms.html}{ExpansionTerms}.
\item
  To change the way feedback documents are selected, implement
  \href{javadoc/org/terrier/querying/FeedbackSelector.html}{FeedbackSelector}.
\end{itemize}

\subsection{Advanced Weighting Models}\label{advanced-weighting-models}

It is very easy to implement your own weighting models in Terrier.
Simply write a new class that extends
\href{javadoc/org/terrier/matching/models/WeightingModel.html}{WeightingModel}.
What's more, there are many examples weighting models in
\href{javadoc/org/terrier/matching/models/package-summary.html}{org.terrier.matching.models}.

\textbf{Generic Divergence From Randomness (DFR) Weighting Models}

The
\href{javadoc/org/terrier/matching/models/DFRWeightingModel.html}{DFRWeightingModel}
class provides an interface for freely combining different components of
the DFR framework. It breaks a DFR weighting model into three
components: the basic model for randomness, the first normalisation by
the after effect, and term frequency normalisation. Details of these
three components can be found from \href{dfr_description.html}{a
description of the DFR framework}. The DFRWeightingModel class provides
an alternate and more flexible way of using the DFR weighting models in
Terrier. For example, to use the
\href{javadoc/org/terrier/matching/models/PL2.html}{PL2} model, the name
of the model \texttt{PL2} should be given in \texttt{etc/trec.models},
or set using the property \texttt{trec.model}. Alternatively, using the
DFRWeightingModel class, we can replace \texttt{PL2} with
\texttt{DFRWeightingModel(P,\ L,\ 2)}, where the three components of PL2
are specified in the brackets, separated by commas. If we do not want to
use one of the three components, for example the first normalisation L,
we can leave the space for this component blank (i.e.
\texttt{DFRWeightingModel(P,\ ,\ 2)}). We can also discard term
frequency normalisation by removing the 2 between the brackets (i.e.
\texttt{DFRWeightingModel(P,\ ,\ )}). However, a basic randomness model
must always be given.

The basic randomness models, the first normalisation methods, and the
term frequency normalisation methods are included in packages
\href{javadoc/org/terrier/matching/models/basicmodel/package-summary.html}{org.terrier.matching.models.basicmodel},
\href{javadoc/org/terrier/matching/models/aftereffect/package-summary.html}{org.terrier.matching.models.aftereffect}
and
\href{javadoc/org/terrier/matching/models/normalisation/package-summary.html}{org.terrier.matching.models.normalisation},
respectively. Many implementations of each are provided, allowing a vast
number of DFR weighting models to be generated.

\subsection{Matching strategies}\label{matching-strategies}

Terrier implements three main alternatives for matching documents for a
given query, each of which implements the
\href{javadoc/org/terrier/matching/Matching.html}{Matching} interface:

\begin{itemize}
\tightlist
\item
  Term-At-A-Time (TAAT) (as per
  \href{javadoc/org/terrier/matching/taat/Full.html}{taat.Full}) -
  exhaustive Matching strategy that scores all postings for a single
  query term, before moving onto the next query term. taat.Full is the
  default Matching strategy for Terrier.
\item
  Document-At-A-Time (DAAT) (as per
  \href{javadoc/org/terrier/matching/daat/Full.html}{daat.Full}) -
  exhaustive Matching strategy that scores all matching query terms for
  a document before moving onto the next documemt. Using daat.Full is
  advantageous for retrieving from large indices.
\item
  \href{javadoc/org/terrier/matching/TRECResultsMatching.html}{TRECResultsMatching}
  - retrieves results from a TREC result file rather than the current
  index, based on the query id. Such a result file must be compatible
  with \href{http://trec.nist.gov/trec_eval}{trec\_eval}.
  TRECResultsMatching can introduce a repeatable efficiency gain for
  batch experiments.
\end{itemize}

If you have a more complex document weighting strategy that cannot be
handled as a
\href{javadoc/org/terrier/matching/models/WeightingModel.html}{WeightingModel}
or
\href{javadoc/org/terrier/matching/dsms/DocumentScoreModifier.html}{DocumentScoreModifier},
you may wish to implement your own Matching strategy. In particular,
\href{javadoc/org/terrier/matching/BaseMatching.html}{BaseMatching} is a
useful base class. Moreover, the
\href{javadoc/org/terrier/matching/PostingListManager.html}{PostingListManager}
should be used for opening the
\href{javadoc/org/terrier/structures/postings/IterablePosting.html}{IterablePosting}
posting stream for each query term.

\subsection{Using Terrier Indices in your own
code}\label{using-terrier-indices-in-your-own-code}

\begin{itemize}
\tightlist
\item
  \textbf{How many documents does term X occur in?}
\item
  \textbf{What is the probability of term Y occurring in the
  collection?}\\
\item
  \textbf{What terms occur in the 11th document?}\\
\item
  \textbf{What documents does term Z occur in?}\\
\end{itemize}

Moreover, if you're not comfortable with using Java, you can dump the
indices of a collection using the --print* options of TrecTerrier. See
the javadoc of
\href{javadoc/org/terrier/applications/TrecTerrier.html}{TrecTerrier}
for more information.

\subsubsection{Example Querying Code}\label{example-querying-code}

Below, you can find a example sample of using the querying
functionalities of Terrier.

\begin{verbatim}
String query = "term1 term2";
SearchRequest srq = queryingManager.newSearchRequest("queryID0", query);
srq.addMatchingModel("Matching", "PL2");
queryingManager.runPreProcessing(srq);
queryingManager.runMatching(srq);
queryingManager.runPostProcessing(srq);
queryingManager.runPostFilters(srq);
ResultSet rs = srq.getResultSet();
\end{verbatim}

{[}\href{extend_indexing.html}{Previous: Extending Indexing}{]}
{[}\href{index.html}{Contents}{]} {[}\href{compression.html}{Next:
Pluggable Compression}{]}

\begin{center}\rule{0.5\linewidth}{\linethickness}\end{center}

Webpage: \url{http://terrier.org}\\
Contact:
\href{mailto:terrier@dcs.gla.ac.uk}{\nolinkurl{terrier@dcs.gla.ac.uk}}\\
\href{http://www.dcs.gla.ac.uk/}{School of Computing Science}\\
Copyright (C) 2004-2015 \href{http://www.gla.ac.uk/}{University of
Glasgow}. All Rights Reserved.
